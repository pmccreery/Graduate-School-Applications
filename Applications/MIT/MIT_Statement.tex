\documentclass[11pt,letterpaper]{article}
\usepackage{fullpage}
\usepackage[top=2cm, bottom=2cm, left=2cm, right=2cm]{geometry}
\usepackage{amsmath,amsthm,amsfonts,amssymb,amscd}
\usepackage{lastpage}
\usepackage{enumerate}
\usepackage{fancyhdr}
\usepackage{mathrsfs}
\usepackage{xcolor}
\usepackage{graphicx}
\usepackage{listings}
\usepackage{hyperref}
\usepackage{float}
\usepackage{gensymb}
\usepackage[utf8]{inputenc}
\usepackage[english]{babel}
\usepackage{setspace}
\pagenumbering{gobble}


% Edit these as appropriate
\newcommand\hwnumber{1}                  % <-- homework number
\newcommand\NetIDa{Patrick McCreery}           % <-- NetID of person #1
\newcommand\NetIDb{108773641}           % <-- NetID of person #2 (Comment this line out for problem sets)
\setlength{\parindent}{2em}
\setlength{\parskip}{1em}
\renewcommand{\baselinestretch}{1}
\pagestyle{fancyplain}
\lhead{\NetIDa}
\rhead{Massachusetts Institute of Technology}
\headsep 2.0em

\begin{document}

At the University of Colorado at Boulder, I am working on an honors thesis project using machine learning to create synthetic images of the solar surface, and use these to train a neural network to perform a Fourier filter. This separates solar granulation from resonant spherical harmonics oscillating on the solar surface without the need for an observationally expensive time-series. The experience of modeling and simulating seemingly random physical phenomena has been one of the most rewarding experiences I have had, and is a driving factor in my desire to enter MIT's PAOC and PPS. This project, which began in 2020, led me to apply to REU programs related to simulations, and I completed my 2021 (delayed from 2020) REU at Montana State University simulating solar flares and loop-top ridge heating post-flare. These research projects have taught me valuable skills in parameter searches and machine learning, for example, but have most importantly helped me understand that modeling is what I wish to pursue in graduate research.
 
My REU experience at Montana State University helped mature my interest in modeling and gave me tangible experience – the results of which will be presented in December at the annual \textbf{AGU conference.} There has been questions about the bright loop-top plasma ridge structures during solar flares, and my work at Montana State had me running simulations to test if compressive magnetosonic shocks with drag could reproduce the observed structure, densities, and magnetic field retraction velocities. My work cast doubt on this idea, opening the field for alternative solutions or approaches. Throughout the REU, I spent significant time reading papers, scouring the code of the simulations I was running, exploring different parameters that impacted heating after solar flares, and comparing the simulations to existing data. From beginning to end, it curated my interest and applicable skills in modeling while also developing my knowledge of solar atmospheric dynamics. 

Professor Mark Rast (CU Boulder) and I began working on the previously described machine learning based Fourier filtering technique before my REU and is currently ongoing, where I will present my project as an honors thesis in the spring. This project has given me experience in a section of modeling quickly coming to prominence -- \textbf{machine learning}. Learning the statistics involved with modeling and machine learning as well as the possible applications of machine learning has increased both the types and number of projects I could work on. By pursuing a minor in statistics and building my knowledge of machine learning from the ground up, I have learned how to apply well-established knowledge from these fields while also having experience in learning from scratch without a class to guide me. \textbf{It is the experience in machine learning, neural networks, and statistics that I hope to bring to MIT}, along with the lessons learned in my research experiences: presentation skills, writing, independence, question asking, and more.

However, solar physics is not where my heart necessarily lies, and as such I have used these experiences to develop skillsets in modeling while taking undergraduate and graduate courses in planetary interiors, surfaces, atmospheres, formations, dynamics, and more. During these classes, atmospheres has always been the most interesting topic to me due to the fascinating dynamics within an atmosphere, the differences between atmospheres in our solar system, and the vast problems that lie ahead in the field that I hope to tackle. The combination of my desire to study planetary science and modeling has let me to MIT -- the place I believe most suitable to bring together my studies and work, while furthering my development as a scientist.

There are many topics I am interested in with reference to studies of exoplanets and planetary atmospheres, but the combination of the two is the field I hope to research: \textbf{exoplanet atmospheres}. Identifying and understanding exoplanet atmospheric spectral signatures is a difficult, yet critical, part of studies in habitability/characterization of atmospheres. This is something that my graduate research would focus on, especially with the (hopeful) launch of JWST, a critical mission in understanding exoplanet atmospheres. Phosphine has been established as an important possible signature of life in these atmospheres, and \textbf{my goal is to work in this area of modeling what phosphine signatures might look like, how we might use JWST to look for signatures of phosphine, and participate in the search for phosphine signatures in atmospheres via observation.} This is ambitious for a graduate student to do alone; in applying to MIT, this opportunity would allow me to actively participate in research in the area of spectra and characterization of exoplanet atmospheres. Professor Seager’s work in exoplanet atmosphere modeling and Assistant Professor de Wit’s data science work in studying planets and dynamics both combine my interest in topic areas with my own practical experience and provide ample opportunity for growth. I bring experience in modeling and data analysis to MIT’s programs, but most importantly, a passion to develop these interests. To be part of a group that possibly detects life outside of the solar system would be a well-spent career, and \textbf{MIT’s PAOC or PPS would be a magnificent place to begin doing so.} The diversity in strengths of the faculty and research conducted at MIT would be well-suited to my many interests in planetary sciences. I believe I can bring skills and hard work to the table to make my studies in these programs beneficial for myself and the program. I have taken graduate level astronomy and math courses at the University of Colorado at Boulder and have done well, so I believe I am ready for the step to graduate studies full-time.


\textbf{Along with research, it is an equal, if not higher, priority to spend my career in the lecture hall, classroom, and community, helping students of all ages engage with science and technology.} We are in a critical time where science literacy is becoming significantly more important; by engaging students with science that MIT and the world is participating in, we can do our part in ensuring the future of the scientific community, and society as a whole. Underrepresented and underprivileged communities may have few opportunities to be exposed to the science a university or institute conducts, and it’s my goal to do anything in my power to ensure students understand what I have been told throughout my undergraduate career -- science is for everyone and can be done by anyone with curiosity and a drive to learn more about the world around them. My interest in science education research and a desire to help my community has placed teaching as a solidified part of my long-term career goals. MIT’s graduate school and programs would be a critical place for me to learn skills and gain experience in teaching while using the available resources to help the community.

Haystack Observatory’s open houses are a prime opportunity to connect with the community and get people excited about science and the universe -- something any astronomer would take advantage of if given the opportunity. Working with the observatory on outreach initiatives is something I hope to do and I have done with the University of Colorado’s Sommers-Bausch Observatory in coordination with CU STARs (Science Technology and Astronomy Recruits). Furthermore, while I am unsure about the status of the group, the SWAG outreach team appears to do fantastic work bringing telescopes to the community -- a program CU STARs hosted on scales of hundreds of people with over 10 telescopes of all sorts, and a program I will continue in graduate school. These groups are great ways to connect with the community and engage with people of all ages and backgrounds who rarely, or never, have an opportunity to look through a telescope. As a graduate student, it would be my hope that I could transition into the current outreach initiatives MIT supports and continue to build on the experience I have gained in my undergraduate career.

My interest in graduate school is a shared sentiment with nearly everyone in the field: a deep desire to learn more about the universe and to share it with the rest of the world. MIT’s programs will help me learn the skills I need to conduct research on exoplanet atmospheres, while I can, in return, bring the passion, desire to learn, and hard work faculty might want in a graduate student.

Thank you very much for taking your time reading this statement of purpose. I look forward to hearing from you soon and exploring the cosmos!
\end{document}
