\documentclass[11pt,letterpaper]{article}
\usepackage{fullpage}
\usepackage[top=2cm, bottom=2cm, left=1.9cm, right=1.9cm]{geometry}
\usepackage{amsmath,amsthm,amsfonts,amssymb,amscd}
\usepackage{lastpage}
\usepackage{enumerate}
\usepackage{fancyhdr}
\usepackage{mathrsfs}
\usepackage{xcolor}
\usepackage{graphicx}
\usepackage{listings}
\usepackage{float}
\usepackage{gensymb}
\usepackage[utf8]{inputenc}
\usepackage[english]{babel}
\usepackage{setspace}
\pagenumbering{gobble}
\usepackage[colorlinks = true,
            linkcolor = blue,
            urlcolor  = blue,
            citecolor = blue,
            anchorcolor = blue]{hyperref}


% Edit these as appropriate
\newcommand\hwnumber{1}                  % <-- homework number
\newcommand\NetIDa{Patrick McCreery}           % <-- NetID of person #1
\newcommand\NetIDb{108773641}           % <-- NetID of person #2 (Comment this line out for problem sets)
\setlength{\parindent}{2em}
\setlength{\parskip}{1em}
\renewcommand{\baselinestretch}{1}
\pagestyle{fancyplain}
\lhead{\NetIDa}
\rhead{University of Hawai'i Institute for Astronomy}
\headsep 2.0em

\begin{document}

At the University of Colorado at Boulder, I am working on an honors thesis project using machine learning to create synthetic images of the solar surface, and using these to train a neural network to perform a Fourier filter. This separates solar granulation from resonant spherical harmonics oscillating on the solar surface without the need for an observationally expensive time-series. The experience of modeling and simulating seemingly random physical phenomena has been one of the most rewarding experiences I have had, and is a driving factor in my desire to enter the University of Hawai'i's  astronomy graduate program. This project, which began in 2020, led me to apply to REU programs related to simulations, and I completed my 2021 (delayed from 2020) REU at Montana State University simulating solar flares, investigating loop-top ridge heating post-flare, and learning about solar magnetodynamics. These research projects have taught me valuable skills in parameter searches and machine learning, for example, but have most importantly helped me understand that modeling is what I wish to pursue in graduate research.
 
\href{http://solar.physics.montana.edu/www/reu/2021/mccreery/}{My REU experience} at Montana State University helped mature my interest in modeling and gave me tangible experience – the results of which will be presented at the annual \textbf{AGU conference.} There have been questions about the origin of bright loop-top plasma ridge structures during solar flares, and my work at Montana State was running simulations to test if compressive magnetosonic shocks with drag could reproduce the observed structure, densities, and magnetic field retraction velocities. My work cast doubt on this idea, opening the field for alternative solutions or approaches. Throughout the REU, I spent significant time reading papers, scouring the \textbf{IDL} code of the simulations I was running, exploring different parameters that impacted heating after solar flares, and comparing the simulations to existing observations. From beginning to end, it curated my interest and applicable skills in modeling while also developing my knowledge of solar magnetic structure. 

Professor Mark Rast (CU Boulder) and I began working on the previously described machine learning based Fourier filtering project before my REU and is currently ongoing, where I will present my project as an honors thesis in the spring. This project is supported by the Lab of Atmospheric and Space Physics through a scholarship awarded based on the quality of work proposal. More about this scholarship can be found on the \href{https://lasp.colorado.edu/home/about/scholarships-and-fellowships/the-charles-a-barth-scholarship-in-space-research/}{Charles A. Barth Scholarship webpage} (https://lasp.colorado.edu/home/about/scholarships-and-fellowships/the-charles-a-barth-scholarship-in-space-research/). This project has provided me experience in a section of modeling quickly coming to prominence -- \textbf{machine learning}. Learning the statistics involved with modeling and machine learning in \textbf{Python} as well as the possible applications of machine learning has increased both the types and number of projects I could work on. By pursuing a minor in statistics and building my knowledge of machine learning from the ground up, I have learned how to apply well-established knowledge from these fields while also having experience in learning from scratch without a class to guide me. \textbf{It is the experience in machine learning, neural networks, and statistics that I hope to bring to UHIfA}, along with the lessons learned in my research experiences: presentation skills, writing, independence, question asking, and more.

However, solar physics is not where my heart necessarily lies, and as such, I have used these experiences to develop skillsets in modeling while taking undergraduate and graduate courses in planetary interiors, surfaces, atmospheres, formations, dynamics, and more. During these classes, planetary formation and planetary system formation has always been the most interesting topic to me due to the fascinating dynamics within a protoplanetary disk that lead to formation of planets, and the importance of these dynamics for the final state of a planetary system. There are still many questions regarding planetary disks that ALMA, future missions, and modeling can help answer. \textbf{UHIfA's work in both solar magnetic structure and investigating protoplanetary disks combine my topic interests with previous research experience.}

UHIfA's involvement as Co-PI of DKIST is one of my major draws to the university, as DKIST will be pivotal in the research of magnetic fields, and their influence on the solar atmosphere. The future of the field is exciting with such a high resolution telescope, and I am applying to UHIfA to be part of projects analyzing data DKIST provides. Energy release and the heating of the solar atmosphere, as well as the impacts of energy release on Earth, are two areas of study that DKIST will provide valuable data for, and these are the projects I hope to contribute to in graduate studies. Dr. Xudong Sun's work specifically interests me, with the combination of numerical models and observation of solar magnetic storms. My work at Montana State developed my interest in further investigating magnetodynamics and solar atmosphere heating, and the work at UHIfA excites me for the future of the field. Being part of projects analyzing data from a revolutionary telescope that could uncover mysteries of the solar atmosphere 

Beyond solar atmospheric dynamics, UHIfA's work in investigating planet formation piques my interest. The observations from ALMA and SMA probe some of the most fascinating parts of the universe and investigate planet-forming disks that provide critical insights into how planetary systems from. In my studies of planetary formation, the open questions regarding the streaming instability as a mechanism for forming planetesimals and the role of turbulence in planet formation interest me most for graduate study. Dr. Johnathan Williams' work modeling gas surface densities 

\textbf{Along with research, it is an equal, if not higher, priority to spend my career in the lecture hall, classroom, and community, helping students of all ages engage with science and technology.} We are in a critical time where science literacy is becoming significantly more important; by engaging students with science that UHIfA and the world is participating in, we can do our part in ensuring the future of the scientific community, and society as a whole. Underrepresented and underprivileged communities may have few opportunities to be exposed to the science a university or institute conducts, and it’s my goal to do anything in my power to ensure students understand what I have been told throughout my undergraduate career -- science is for everyone and can be done by anyone with curiosity and a drive to learn more about the world around them. My interest in science education research and a desire to help my community has placed teaching as a solidified part of my long-term career goals. UHIfA’s program would be a critical place for me to learn skills and gain experience in teaching while using the available resources to help the community.

UHIfA currently hosts similar outreach initiatives as the astronomy department at CU Boulder, which is another point of interest in the program. Programs like the AWI and the Maunakea Scholars Program are important for the vitality of science, and are the types of initiatives that I hope to work with in graduate study. In my undergraduate experience, I have worked with the University of Colorado’s Sommers-Bausch Observatory (SBO) and CU STARs (Science Technology and Astronomy Recruits). For SBO, I helped host virtual open-houses during the pandemic, and in the spring semester, I will be trained on the main 24-inch telescopes to assist public open house visits. The goal of CU STARs, as explained on their website, is to effectively bring the stars to the people. We plan classroom visits and curate class lessons for students of all ages in rural Colorado in the hopes of engaging underrepresented students with the University's work. UHIfA hosts star parties and does classroom visits, both of which CU STARs has done. The star parties that I helped run hosted hundreds of people of all ages and backgrounds with over 10 telescopes of different kinds. These experiences have been the most rewarding to me, and participating in UHIfA's outreach initiatives would be a priority in my graduate studies. These programs UHIfA have are great ways to connect with the community and engage with people of all ages and backgrounds who rarely, or never, have an opportunity to look through a telescope. \textbf{As a graduate student, it would be my hope that I could transition into the current outreach initiatives UHIfA supports and continue to build on the experience I have gained in my undergraduate career.}

My interest in graduate school is a shared sentiment with nearly everyone in the field: a deep desire to learn more about the universe and to share it with the rest of the world. UHIfA’s programs will help me learn the skills I need to conduct research on solar magnetic structure and protoplanetary disks, while I can, in return, bring the passion, desire to learn, and hard work faculty might want in a graduate student.

Thank you very much for taking your time reading this statement. I look forward to hearing from you soon and exploring the cosmos!
\end{document}
