\documentclass[11pt,letterpaper]{article}
\usepackage{fullpage}
\usepackage[top=2cm, bottom=2cm, left=2cm, right=2cm]{geometry}
\usepackage{amsmath,amsthm,amsfonts,amssymb,amscd}
\usepackage{lastpage}
\usepackage{enumerate}
\usepackage{fancyhdr}
\usepackage{mathrsfs}
\usepackage{xcolor}
\usepackage{graphicx}
\usepackage{listings}
\usepackage{hyperref}
\usepackage{float}
\usepackage{gensymb}
\usepackage[utf8]{inputenc}
\usepackage[english]{babel}
\usepackage{setspace}
\pagenumbering{gobble}


% Edit these as appropriate
\newcommand\hwnumber{1}                  % <-- homework number
\newcommand\NetIDa{Patrick McCreery}           % <-- NetID of person #1
\newcommand\NetIDb{108773641}           % <-- NetID of person #2 (Comment this line out for problem sets)
\setlength{\parindent}{2em}
\setlength{\parskip}{1em}
\renewcommand{\baselinestretch}{1}
\pagestyle{fancyplain}
\lhead{\NetIDa}
\rhead{University of Minnesota}
\headsep 2.0em

\begin{document}

As a graduate student, I would like to make significant strides in actively helping underrepresented and underprivileged students in STEM realize their potential and further every student's curiosity about the world around them. It is well documented that astronomy and physics have some of the poorest gender and race representation of any sciences, and this can be aided by encouraging and fostering curiosity in physics/astronomy among \textit{all} students, which is something that I hope to do in graduate school and beyond. As a graduate student part of a major university, I would be in a prime position to help students who are less fortunate, whether that be from systematic discrimination, unfortunate circumstances, poor school funding, or many other possible circumstances. UMN's outreach efforts, such as public observing nights (when these return), Bell Museum's youth and K-12 outreach initiatives, and the planetarium's programs are all efforts to bring science to the community that I would like to participate in. Each of these programs/organizations have the opportunity to reach underrepresented students, engage students in science, and promote diversity in the field. Engaging students with science is the key to promoting diversity. Telescopes, for example, foster an interest in the universe, and can prompt students to earn an astronomy degree. However, the price of telescopes also prices demographics out of this, meaning a university like UMN can step in with programs to promote engagement in science to underrepresented and underprivileged students, which would promote diversity in the field. Diversity, however, isn't just about numbers on a fact-sheet; the vitality of science as a whole depends on a diverse group of people to bring novel ideas and approaches to existing problems.

Attending CU Boulder, I met Professor Erica Ellingson, who invited me to help CU STARs (Science, Technology, and Astronomy Recruits) in their goal to bring the sky to underrepresented students in STEM. CU STARs is a group of passionate graduate and undergraduate students participating in a plethora of activities aimed at the goal of bringing astronomy, and STEM more broadly, to underrepresented groups who would typically not have the opportunity to learn about the sky or realize possible research opportunities available to them. My personal participation in CU STARs was mostly two-fold -- helping with campus-wide events and helping create and present class lessons to be given to students of many different educational backgrounds. As an example, we put together a class lecture on black holes, with varying levels of complexities depending on the level of the K-12 students. Our efforts were directed towards high schools and middle schools with different objectives and lessons at each level. The overall goal, however, was to engage students with material and encourage them to stay engaged with their school, their broader community, and at the university level studying physics/astronomy in the case of high school students.

This is the type of outreach and educational work that I plan on continuing in my graduate education, and UMN is doing fantastic work I would like to participate in. Campus-wide star parties brought hundreds of students (pre-pandemic) from every background and demographic to a wide variety of telescopes we set up to help them get a taste of the universe. It’s the experience of showing the universe to people who have never looked through a telescope that I want to continue to bring to university and K-12 students, especially those in underrepresented communities. Viewing the night sky is a powerful engagement tool that is not only easy to present, but powerful to interest students who may have never considered a career in STEM or academia.

To engage students, it's often important to not only provide interesting material at an understandable level, but to make it relatable so that they can see themselves conducting research. This is what the fields of astronomy and astrophysics miss, and as a graduate student I hope to reach students by providing class lessons to science courses in rural or poorly-funded schools, where a lesson on constellations, black holes, or telescopes can engage students with astronomy. Graduate students are, in general, relatively young, and are in a unique position where students may see them as more relatable and engage more effectively. Ultimately what will drive students to pursue STEM degrees (and boost diversity in the field), irregardless of background, is their ability to see themselves participating in research.


\end{document}
