\documentclass[11pt,letterpaper]{article}
\usepackage{fullpage}
\usepackage[top=2cm, bottom=3cm, left=1.5cm, right=1.5cm]{geometry}
\usepackage{amsmath,amsthm,amsfonts,amssymb,amscd}
\usepackage{lastpage}
\usepackage{enumerate}
\usepackage{fancyhdr}
\usepackage{mathrsfs}
\usepackage{xcolor}
\usepackage{graphicx}
\usepackage{listings}
\usepackage{float}
\usepackage{gensymb}
\usepackage[utf8]{inputenc}
\usepackage[english]{babel}
\usepackage{setspace}
\pagenumbering{gobble}
\usepackage[colorlinks = true,
            linkcolor = blue,
            urlcolor  = blue,
            citecolor = blue,
            anchorcolor = blue]{hyperref}

% Edit these as appropriate
\newcommand\hwnumber{1}                  % <-- homework number
\newcommand\NetIDa{Patrick McCreery}           % <-- NetID of person #1
\newcommand\NetIDb{108773641}           % <-- NetID of person #2 (Comment this line out for problem sets)
\setlength{\parindent}{2em}
\setlength{\parskip}{1em}
\renewcommand{\baselinestretch}{1}
\pagestyle{fancyplain}
\lhead{\NetIDa}
\rhead{University of Minnesota}
\headsep 2.0em

\begin{document}

At the University of Colorado at Boulder, I am working on an honors thesis project using machine learning to create synthetic images of the solar surface, and using these to train a neural network to perform a Fourier filter. This separates solar granulation from resonant spherical harmonics oscillating on the solar surface without the need for an observationally expensive time-series. The experience of modeling and simulating seemingly random physical phenomena has been one of the most rewarding experiences I have had, and is a driving factor in my desire to enter the University of Minnesota's physics and astronomy graduate program. This project, which began in 2020, led me to apply to REU programs related to simulations, and \textbf{I completed my 2021 (delayed from 2020) REU at Montana State University simulating solar flares}, investigating loop-top ridge heating post-flare, and learning about solar magnetodynamics. These research projects have taught me valuable skills in parameter searches and machine learning, for example, but have most importantly helped me understand that modeling and solar physics is what I wish to pursue in graduate research.
 
Professor Mark Rast (CU Boulder) and I began working on the previously described machine learning based Fourier filtering project before my REU and is currently ongoing, where I will present my project as an honors thesis in the spring. This project is supported by the Lab of Atmospheric and Space Physics through a scholarship awarded based on the quality of work proposal. More about this scholarship can be found on the \href{https://lasp.colorado.edu/home/about/scholarships-and-fellowships/the-charles-a-barth-scholarship-in-space-research/}{Charles A. Barth Scholarship webpage}. This project has provided me experience in a section of modeling quickly coming to prominence -- \textbf{machine learning}. Learning the statistics involved with modeling and machine learning in \textbf{Python} as well as the possible applications of machine learning has increased both the types and number of projects I could work on. By pursuing a minor in statistics and building my knowledge of machine learning from the ground up, I have learned how to apply well-established knowledge from these fields while also having experience in learning from scratch without a class to guide me. \textbf{It is the experience in machine learning, neural networks, and statistics that I hope to bring to MIfA}, along with the lessons learned in my research experiences: presentation skills, writing, independence, question asking, and more.

\href{http://solar.physics.montana.edu/www/reu/2021/mccreery/}{My REU experience} at Montana State University helped mature my interest in modeling and gave me tangible experience – the results of which will be presented at the annual \textbf{AGU conference.} There have been questions about the origin of bright loop-top plasma ridge structures during solar flares, and my work at Montana State was running simulations to test if compressive magnetosonic shocks with drag could reproduce the observed structure, densities, and magnetic field retraction velocities. My work cast doubt on this idea, opening the field for alternative solutions or approaches. Throughout the REU, I spent significant time reading papers, scouring the \textbf{IDL} code of the simulations I was running, exploring different parameters that impacted heating after solar flares, and comparing the simulations to existing observations. From beginning to end, it curated my interest and applicable skills in modeling while also developing my knowledge of solar magnetic structure. 

This REU curated my interest in solar physics, and specifically, the dynamics of solar flare, CMEs, and solar weather in general. The underlying magnetohydrodynamic equations that drive the dynamics of solar weather are interesting, but comparing simulated phenomena to observation is extremely rewarding and something my graduate studies would revolve around. Building off of this, Dr. Glesener's work interests me, specifically using EUV and hard X-rays to investigate high-energy emissions from flares. New instruments are critical in developing our understanding of high-energy events in the sun, and working with Dr. Glesener would be ideal to develop my understanding of energy release via solar flares or CMEs, while also working on developing missions to help with these observations.

\pagebreak

Dr. Qian's work also interests me, specifically about the formation of the solar system. In my coursework, I have taken planetary science courses that have developed my interest beyond solar physics and into planetary science. The fundamental question of how our solar system formed is important to understand the current layout of the solar system, as well as systems beyond our own. Furthermore, Dr. Qian's work on supernovae and their impacts on element creation via the r-process interests me, as understanding element creation is fundamental to understand the universe, but also to how solar systems can form. The creation of elements in supernovae is important to study for understanding the clouds future generation stars will be built from, as metallicity is an important for solar system properties.

One of the most intriguing aspects of UMN's program is the DSMMA program. It took until my junior year with classes in data analysis in astronomy, a probability course, and a statistics course to realize I was deeply interested in data science -- one of the fastest growing areas in and out of academia. Unfortunately, I did not have the schedule space to add a statistics degree, but I have added an applied statistics minor. Astronomy is collecting more and more data to probe more and more complex problems in all areas of study via observation, and data science is becoming a critical aspect of being an astronomer. The applicability of data science to astronomy, and the world in general, make UMN's program one of the best for my career preparation to use large data sets for research applications in astronomy.

Throughout my undergraduate career, I have been taking courses at the graduate and undergraduate level to prepare myself for the physics and mathematics that will come with entering a graduate program. I have taken graduate level applied partial differential equations, as well as graduate level planetary surfaces and interiors, both of which I received an A in. These courses have taught me lessons in the mentality shift between undergraduate and graduate level studies, preparing me for this shift to a graduate program. Because of this, I am confident in my ability to adapt to graduate studies, but I am also prepared to take on the more difficult math and physics in a graduate program. I have taken a full year of classical mechanics, electrodynamics, and quantum mechanics, meaning I am confident in succeeding at UMN. I understand it will be difficult and take perseverance, but the graduate courses have helped me understand that I want to give my best effort.

\textbf{Along with research, it is an equal, if not higher, priority to spend my career in the lecture hall, classroom, and community, helping students of all ages engage with science and technology.} We are in a critical time where science literacy is becoming significantly more important; by engaging students with science that MIfA and the world is participating in, we can do our part in ensuring the future of the scientific community, and society as a whole. Underrepresented and underprivileged communities may have few opportunities to be exposed to the science a university or institute conducts, and it’s my goal to do anything in my power to ensure students understand what I have been told throughout my undergraduate career -- science is for everyone and can be done by anyone with curiosity and a drive to learn more about the world around them. My interest in science education research and a desire to help my community has placed teaching as a solidified part of my long-term career goals. MIfA’s program would be a critical place for me to learn skills and gain experience in teaching while using the available resources to help the community.

My interest in graduate school is a shared sentiment with nearly everyone in the field: a deep desire to learn more about the universe and to share it with the rest of the world. MIfA’s programs will help me learn the skills I need to conduct research on solar magnetic structure and high-energy physics in the solar atmosphere, while I can, in return, bring the passion, desire to learn, and hard work faculty might want in a graduate student. 

Thank you very much for taking your time reading this statement. I look forward to hearing from you soon and exploring the cosmos!

\end{document}
