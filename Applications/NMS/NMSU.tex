\documentclass[12pt,letterpaper]{article}
\usepackage{fullpage}
\usepackage[top=2cm, bottom=2cm, left=1cm, right=1cm]{geometry}
\usepackage{amsmath,amsthm,amsfonts,amssymb,amscd}
\usepackage{lastpage}
\usepackage{enumerate}
\usepackage{fancyhdr}
\usepackage{mathrsfs}
\usepackage{xcolor}
\usepackage{graphicx}
\usepackage{listings}
\usepackage{hyperref}
\usepackage{float}
\usepackage{gensymb}
\usepackage[utf8]{inputenc}
\usepackage[english]{babel}
\usepackage{setspace}
\pagenumbering{gobble}


% Edit these as appropriate
\newcommand\hwnumber{1}                  % <-- homework number
\newcommand\NetIDa{Patrick McCreery}           % <-- NetID of person #1
\newcommand\NetIDb{108773641}           % <-- NetID of person #2 (Comment this line out for problem sets)
\setlength{\parindent}{2em}
\setlength{\parskip}{1em}
\renewcommand{\baselinestretch}{1.9}
\pagestyle{fancyplain}
\lhead{\NetIDa}
\rhead{New Mexico State University}
\headsep 2.0em

\begin{document}

At the University of Colorado at Boulder, I am working on an honors thesis project using machine learning to create synthetic images of the solar surface, and use these to train a neural network to perform a Fourier filter. This separates solar granulation from resonant spherical harmonics oscillating on the solar surface without the need for an observationally expensive time-series. The experience of modeling and simulating seemingly random physical phenomena has been one of the most rewarding experiences I have had, and is a driving factor in my desire to enter NMSU's astronomy graduate program. This project, which began in 2020, led me to apply to REU programs related to simulations, and I completed my 2021 (delayed from 2020) REU at Montana State University simulating solar flares and loop-top ridge heating post-flare. These projects helped me understand that I would like a career in modeling stellar or planetary phenomena. 
 
My REU experience at Montana State University helped mature my interest in modeling and gave me tangible experience – the results of which were presented in December at the annual \textbf{AGU conference.} There have been questions about the bright loop-top plasma ridge structures during solar flares, and my work at Montana State involved running simulations to test if compressive magnetosonic shocks with drag could reproduce the observed structure, densities, and magnetic field retraction velocities. My work cast doubt on this idea, opening the field for alternative solutions or approaches. The project curated my interest and applicable skills in modeling while also developing my knowledge of solar atmospheric dynamics. 

Professor Mark Rast (CU Boulder) and I began working on the previously described machine learning based Fourier filtering technique before my REU and is currently ongoing, where I will present my project as an honors thesis in the spring. This project has given me experience in a section of modeling quickly coming to prominence -- \textbf{machine learning}. Learning the statistics involved with modeling and machine learning as well as the possible applications of machine learning has increased both the types and number of projects I could work on. By pursuing a minor in statistics and building my knowledge of machine learning from the ground up, I have learned how to apply well-established knowledge from these fields while also having experience in learning from scratch without a class to guide me. \textbf{It is the experience in machine learning, neural networks, and statistics that I hope to bring to and develop at NMSU}.

However, along with solar physics, I have used courses to broaden and develop my understanding of planetary science subjects likes planetary interiors, surfaces, atmospheres, formations, dynamics, and more. I have taken a graduate course in planetary surfaces and interiors to further prepare myself for the level of study and to better my knowledge in the field. 

Regarding research, I am interested in Assistant Professor Wladimir Lyra's and Professor Jason Jackiewicz's work. Dr. Lyra's work interests me because of their use of simulations to understand planet formation. There are many questions regarding planet formation that I would like to study in my career, including inconsistencies in the Nice and Grand Tack model, the role of streaming instabilities/turbulence/pebble accretion in planetesimal formation, and the presence/importance of the Late Heavy Bombardment. Professor Jackiewicz's work particularly interests me because it combines the topics I have been working on in my research projects already. Helioseismology helps answer important questions regarding stellar structure, and my interest in the subject motivated me to apply to the graduate program, as it is an area of research I hope to contribute to in graduate school and beyond. 

\textbf{Along with research, it is an equal, if not higher, priority to spend my career in the lecture hall, classroom, and community, helping students of all ages engage with science and technology.} Along with lectures and lessons for students, observatory open houses are a prime opportunity to connect with the community and get people excited about science and the universe -- something any astronomer would take advantage of if given the opportunity. Working with the observatory on outreach initiatives is something I hope to do and I have done with the University of Colorado’s Sommers-Bausch Observatory in coordination with CU STARs (Science Technology and Astronomy Recruits). As a graduate student, it would be my hope that I could transition into the current outreach initiatives NMSU supports and continue to build on the experience I have gained in my undergraduate career.


\end{document}
