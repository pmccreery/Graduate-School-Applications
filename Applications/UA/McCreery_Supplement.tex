\documentclass[11pt,letterpaper]{article}
\usepackage{fullpage}
\usepackage[top=2cm, bottom=2cm, left=2cm, right=2cm]{geometry}
\usepackage{amsmath,amsthm,amsfonts,amssymb,amscd}
\usepackage{lastpage}
\usepackage{enumerate}
\usepackage{fancyhdr}
\usepackage{mathrsfs}
\usepackage{xcolor}
\usepackage{graphicx}
\usepackage{listings}
\usepackage{hyperref}
\usepackage{float}
\usepackage{gensymb}
\usepackage[utf8]{inputenc}
\usepackage[english]{babel}
\usepackage{setspace}
\pagenumbering{gobble}


% Edit these as appropriate
\newcommand\hwnumber{1}                  % <-- homework number
\newcommand\NetIDa{Patrick McCreery}           % <-- NetID of person #1
\newcommand\NetIDb{108773641}           % <-- NetID of person #2 (Comment this line out for problem sets)
\setlength{\parindent}{2em}
\setlength{\parskip}{1em}
\renewcommand{\baselinestretch}{1}
\pagestyle{fancyplain}
\lhead{\NetIDa}
\rhead{University of Arizona}
\headsep 2.0em

\begin{document}
As a graduate student, I would like to make significant strides in actively helping underrepresented and underprivileged students in STEM realize their potential and further every student's curiosity about the world around them. It is well documented that astronomy and physics have some of the poorest gender and race representation of any sciences, and this can be aided by encouraging and fostering curiosity in physics/astronomy among \textit{all} students, which is something that I hope to do in graduate school and beyond. Recognizing the problem is one step, but actively solving the problem is another, and I believe I could be in a prime position to take strides in solving the problem if admitted. It is the duty of graduate students and those in more senior positions, who have been fortunate enough to be part of the scientific community, to help students who are less fortunate, whether that be from systematic discrimination, unfortunate circumstances, poor school funding, or many other possible circumstances. Steward Observatory's work, with programs like Astronomy Camp and Sky School, is in the outreach area that I would like to continue participating in as a graduate student.

The K-12 district I attended for my whole life had a school board president who was a rocket scientist -- meaning I had a plethora of opportunities to get involved in STEM opportunities. These experiences, like an advanced calculus course that was only taught because we could pay a qualified teacher to teach it, were important to developing my curiosity and interest in a STEM career that others may not have the luxury of being able to have. However, even with plenty of access to quality STEM courses, I still was not exactly sold that I had what it takes to participate in research. This was because I couldn't see myself in the field; I had a conception of what a scientist was, and I didn't see myself in that image. This is what I believe can help alleviate some of the demographic issues in astronomy. There are people of all types active in all types of research, and I hope to do my part to show that science isn't a black box and research doesn't mean sitting in an office, writing on a chalk board. 

Attending university in Boulder, Colorado I met Professor Erica Ellingson, who invited me to help CU STARs (Science, Technology, and Astronomy Recruits) in their goal to bring the sky to underrepresented students in STEM. I joined because I understood the opportunities that others may not have, and I wanted to help extend opportunities to students who may not see themselves in STEM. CU STARs’ is a group of passionate graduate and undergraduate students participating in a plethora of activities aimed at the goal of bringing astronomy, and STEM more broadly, to underrepresented groups who would typically not have the opportunity to learn about the sky or realize possible research opportunities available to them. 

My personal participation in CU STARs was mostly two-fold -- helping with campus events and helping create and present class lessons to be given to students of many different educational backgrounds. As an example, we put together an in-class lecture on black holes, with varying levels of complexities depending on the level of the students. Our efforts were directed towards high schools and middle schools with different objectives and lessons at each level. The overall goal, however, was to engage students with material and encourage them to stay engaged with their school, their broader community, and at the university level studying physics/astronomy in the case of high school students.

This is the type of outreach and educational work that I plan on continuing in my graduate education, and  Steward Observatory is doing fantastic work I would like to participate in. The rewarding experience of engaging students who haven’t had significant exposure to academia is something that not only motivates me to better the community and my work, but motivates others to do the same. One person can't solve the problem; it takes the whole community. The only way we push science forward is through many perspectives and many skeptical eyes, so increasing the visibility of STEM academia isn’t something to check a box or to make people feel warm that their program is “representative,” but rather a critical necessity of humanity’s endeavor to understand the universe. 

\pagebreak

Part of my work with STARs -- and something that I hope to continue doing in graduate school -- is to bring the stars to people. Students in other fields of study, such as economics or the humanities, have the curiosity to understand the night sky, but perhaps not the guidance to know how to achieve that understanding. Campus-wide star parties brought hundreds of students (pre-pandemic) from every background and demographic to a wide variety of telescopes we set up to help them get a taste of the universe. The Space Minor at CU, with the man-power of CU STARs, funded these campaigns to bring business, engineering, arts, humanities, etc. majors to the minor, which brings science literacy and the stars to those on campus who do not have the schedule space or prerequisites to take on a whole new major of astrophysics, yet still want to learn.

Helping set-up and use the telescopes -- such as my favorite, the well named “Bad Boy” Meade LX90 -- was rewarding and always reinvigorated my passion for astronomy. Students seeing the bands of Jupiter, the Galilean satellites, or Saturn’s rings for the first time is enjoyable, and an important step in helping teach that the stars are for anyone’s enjoyment -- not just astrophysics majors, PhDs, or professors. It’s these experiences that I want to continue to bring to university and K-12 students, especially those in underrepresented communities and non-STEM majors. Viewing the night sky is a powerful engagement tool that is not only easy to present, but powerful to interest students who may have never considered a career in STEM or academia. It doesn’t take a lab coat or a chalkboard to do science, but anyone can do it in their backyard or local observatory/astronomy club.  

To engage students, it's often important to not only provide interesting material at an understandable level, but to make it relatable so that they can see themselves conducting research. This is what the fields of astronomy and astrophysics miss, and as a graduate student I hope to reach students by providing class lessons to science courses in rural or poorly-funded schools, where a lesson on constellations, black holes, or telescopes can engage students with astronomy. Graduate students are, in general, relatively young, and are in a unique position where students may see them as more relatable and engage more effectively. Ultimately what will drive students to pursue STEM degrees is their ability to see themselves participating in research, which can be done by bringing material to the students and show that regular people of all backgrounds are conducting research, not just people of one demographic who sit in offices all day writing on a chalkboard.

It’s critical for graduate students to understand their ability to help students in their community engage with science, even if it may not lead to a career in science for that student. Graduate students have the capacity to make impactful contributions in their communities and demonstrate to students that they are capable of participating in research and furthering their education. There were many times in my K-12 education where I was unsure if I had what it took to become a scientist, or if I even wanted to. I took my astronomy class with the wrestling coach who was contractually obligated to teach the course, while I knew there were students in private schools learning from educators with PhDs. If I had the opportunity to listen or talk to a graduate student in my time in middle or high school, it would have gone a long way to understanding that I had a place in science. This was only my experience as someone who comes from a good school district and was able to eventually see myself in a place to potentially do science, and as a graduate student, I want to ensure I can help students who were not as fortunate as me to understand that anyone with the curiosity and determination to succeed can fit into the scientific community. At best, students pursue a career in STEM, and at worst, we better the scientific literacy of our community.

\textit{“Every kid starts out as a natural-born scientist, and then we beat it out of them. A few trickle through the system with their wonder and enthusiasm for science intact.”}

- Carl Sagan
\end{document}
