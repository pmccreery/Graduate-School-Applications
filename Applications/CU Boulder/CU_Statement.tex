\documentclass[11pt,letterpaper]{article}
\usepackage{fullpage}
\usepackage[top=2cm, bottom=2cm, left=2cm, right=2cm]{geometry}
\usepackage{amsmath,amsthm,amsfonts,amssymb,amscd}
\usepackage{lastpage}
\usepackage{fancyhdr}
\usepackage{xcolor}
\usepackage{float}
\usepackage{gensymb}
\usepackage[utf8]{inputenc}
\usepackage[english]{babel}
\pagenumbering{gobble}



% Edit these as appropriate
\newcommand\hwnumber{1}                  % <-- homework number
\newcommand\NetIDa{Patrick McCreery}           % <-- NetID of person #1
\newcommand\NetIDb{108773641}           % <-- NetID of person #2 (Comment this line out for problem sets)
\setlength{\parindent}{1em}
\setlength{\parskip}{1em}
\renewcommand{\baselinestretch}{1}
\pagestyle{fancyplain}
\lhead{\NetIDa}
\rhead{University of Colorado at Boulder}
\headsep 2em

\begin{document}


Astronomy as a science is a unique human endeavor — bringing together the hard science of observation with the theory of philosophy. Astronomy not only probes the possibilities of the universe, but tests our resolve in understanding the consequences of being alive, exploring the universe, and considering if the universe is something to explore or something to observe from our backyard. This combination has pushed me to earn an undergraduate degree in astronomy/astrophysics and now a graduate degree. I’m driven to further our understanding of the great unknown, but also intrigued by the philosophical discussions to be had about exploring the universe and possible consequences of discoveries -- like the possible habitability of other solar systems. My understanding of the universe has breadth because of my undergraduate degree; however, now I look forward to earning a graduate degree and gaining depth in understanding of the topics I find most interesting by using the skillsets and knowledge I have gathered in my undergraduate experience. The topics that most interest me lie primarily in solar physics and planetary science. 
 
Modeling of the solar interior, as discussed in introductory astrophysics courses, sparked my interest in solar physics; combining theory with modeling techniques in an attempt to understand a difficult topic like interior stellar structure was fascinating to me, and led me to look for work in solar physics. My continued interest in solar physics and modeling guided me to 2 major research projects -- one at the University of Colorado at Boulder with Professor Mark Rast, and one at Montana State University with Professor Dana Longcope as part of my REU program. My project at Montana State involved the modeling of heating via magnetic reconnection after a solar flare, which required an understanding of the magnetohydrodynamic equations governing compressive shocks. We proposed these shocks as the method of heating loop-top ridges in a two-ribbon flare. My current work with Professor Rast involves using machine learning to perform a Fourier filter without the necessity of an observationally expensive time-series. Specifically, we are applying this to images of the Sun to separate resonant acoustic modes from background solar granulation.

Although solar physics is my primary interest of study, outer solar system moons are another point of interest because of the mystery and possibilities of subsurface oceans -- we’re beginning to look into places we once never thought possible, and I hope to be part of this developing area. In freshman-level astronomy, I was astounded by the satellites of Jupiter and Saturn, specifically the variety in surface features. Io being the most volcanically active body in the solar system and the plumes of Europa and Enceladus caught my attention, as these bodies far away have many features and mysteries to uncover -- something I hope to be a part of. This interest guided me through many courses, up until graduate level planetary surfaces and interiors course, where I am now working on a class project in modeling surface deposition of material from plumes on Europa using known information about Enceladus. This is the type of work I hope to continue in graduate school - modeling and predicting processes for different solar system bodies, then comparing them to data we have, or might have with upcoming missions such as E-THEMIS.

The University of Colorado at Boulder would be the ideal place for my studies due to the varied strengths and areas of study of the faculty. Being in the department for four years, I have gotten to know many faculty and their work, which is why I am applying to return. I know this department is right for me and my professional development, and I can help contribute to the department's research and goals. Professor Mark Rast and Asst. Professor Ben Brown are conducting work in fluid dynamics, solar surface features, and convection that I am interested in working on. NSO has a plethora of researchers I would be interested in working, and I am especially excited to work with DKIST data that will be revolutionizing in solar physics. Asst. Professor Paul Hayne is researching icy satellites, which is an area I have been working to become involved in; Asst. Professor Meredith MacGregor is working on planetary system formation and habitability, which is another topic I've recently found myself quite interested in that mixes my interests in solar and planetary science. Specifically, I find Dr. MacGregor's modeling of debris disks with ALMA observations to be interesting. 
 
\pagebreak
 
Pivoting away from my interests and into graduate preparation, my REU experience at Montana State University matured my experience with modeling from a topic of interest to an acquired skill set along with tangible experience – the results of which will be presented in December at the annual AGU conference. Throughout the REU, I spent significant time reading papers, the code of the simulations I was running, exploring different parameters that impacted heating, and comparing simulation results to data we have from solar flares. From the beginning to end, it curated my interest and applicable skills in modeling, while also developing my knowledge of the solar atmosphere. My undergraduate project with Professor Mark Rast began before my REU, and is currently ongoing, where I will present my project as an honors thesis in the spring. This project has given me experience in a section of modeling quickly coming to prominence -- machine learning. Learning the statistics involved with modeling and machine learning via my minor in statistics, as well as the possible applications of machine learning has opened the types, and number, of projects I could work on. Beyond machine learning, however, I am learning about helioseismology as a method of understanding the solar interior and applying this knowledge, along with Fourier analysis, to practical problems. Combining my interest in machine learning and neural networks to perform modeling and simulations is an area I not only hope to learn more about, but to work on.
   
Research experience is only half of the necessary experience, however. Graduate study requires more than just an interest and research experience -- it requires organization, time management, teamwork, and a certain amount of flexibility. I have already proved to myself that I am capable of graduate study, as I have taken multiple graduate level courses in astronomy and mathematics. The research projects I have been on, along with a plethora of course projects, has given skills to work on teams, in research groups, and an understanding of the necessary work ethic in academia. The 10-week research experience I had at my REU taught me valuable lessons in getting presentable work done in just 10 weeks. I purposefully took a graduate level partial differential equations course, took classes for a statistics minor (two of which are undergraduate/graduate level), extra physics courses (higher level quantum mechanics), a research methods course, and a graduate level astronomy course for the purpose of preparing myself for the graduate level experience and being able to hit the ground running in the fall with classes and work that I’m passionate about. In the graduate course I have taken at CU, I have not missed a step in the acceleration of the educational experience, which proves I am ready for the step to becoming a full-time graduate student at CU.
  
Along with research, it is an equal, if not higher, priority to spend my career in the lecture hall, classroom, and community, helping students of all ages engage with science and technology. We are in a critical time where science literacy is becoming significantly more important; by engaging students with science that CU and the world is participating in, we can do our part in ensuring the future of the scientific community, and society as a whole. Underrepresented and underprivileged communities may have few opportunities to be exposed to the science a university or institute conducts, and it’s my goal to do anything in my power to ensure students understand what I have been told throughout my undergraduate career -- science is for everyone and can be done by anyone with curiosity and a drive to learn more about the world around them. CU has a plethora of programs that would help me develop in this area, with CU STARs (I am currently part of), which is sadly coming to an end, but only in name, Fiske Planetarium's outreach initiatives, and the APS department's shared desire to increase the visibility of science to underrepresented communities.
  
In the end, my interest in graduate school is a shared sentiment with nearly everyone in the field: a deep desire to learn more about the universe and to share it with the rest of the world. I believe CU Boulder would be the right place for me to continue my studies and develop myself as a scientist. In return, I hope to contribute to the work the department is conducting, as well as give back to the community.

Thank you for taking the time to read my personal statement; I look forward to hearing from you soon and exploring the cosmos!
\end{document}
