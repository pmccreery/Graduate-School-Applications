\documentclass[11pt,letterpaper]{article}
\usepackage{fullpage}
\usepackage[top=2cm, bottom=2.5cm, left=2.5cm, right=2cm]{geometry}
\usepackage{amsmath,amsthm,amsfonts,amssymb,amscd}
\usepackage{lastpage}
\usepackage{enumerate}
\usepackage{fancyhdr}
\usepackage{mathrsfs}
\usepackage{xcolor}
\usepackage{graphicx}
\usepackage{listings}
\usepackage{hyperref}
\usepackage{float}
\usepackage{gensymb}
\usepackage[utf8]{inputenc}
\usepackage[english]{babel}
\usepackage{setspace}
\pagenumbering{gobble}


% Edit these as appropriate
\newcommand\hwnumber{1}                  % <-- homework number
\newcommand\NetIDa{Patrick McCreery}           % <-- NetID of person #1
\newcommand\NetIDb{108773641}           % <-- NetID of person #2 (Comment this line out for problem sets)
\setlength{\parindent}{2em}
\setlength{\parskip}{1em}
\renewcommand{\baselinestretch}{1}
\pagestyle{fancyplain}
\lhead{\NetIDa}
\rhead{Montana State University}
\headsep 2.0em

\begin{document}

At the University of Colorado at Boulder, I am working on an honors thesis project using machine learning to create synthetic images of the solar surface, and use these to train a neural network to perform a Fourier filter. This separates solar granulation from resonant spherical harmonics oscillating on the solar surface without the need for an observationally expensive time-series. The experience of modeling and simulating seemingly random physical phenomena has been one of the most rewarding experiences I have had, and is a driving factor in my desire to enter MSU's solar physics program. This project, which began in 2020, led me to apply to REU programs related to simulations, and I completed my 2021 (delayed from 2020) REU here at Montana State University simulating solar flares and loop-top ridge heating post-flare. These research projects have taught me valuable skills in parameter searches and machine learning, for example, but have most importantly helped me understand that modeling is what I wish to pursue in graduate research. My REU experience also helped me understand that Montana State University is the ideal place for my graduate studies. 
 
My REU experience at Montana State University helped mature my interest in modeling and gave me tangible experience – the results of which were presented in December at the annual \textbf{AGU conference.} There have been questions about the bright loop-top plasma ridge structures during solar flares, and my work at Montana State involved running simulations to test if compressive magnetosonic shocks with drag could reproduce the observed structure, densities, and magnetic field retraction velocities. My work cast some doubt on this idea, opening the field for alternative solutions, approaches, and modifications to my work. Throughout the REU, I spent significant time reading papers, scouring the code of the simulations I was running, exploring different parameters that impacted heating after solar flares, and comparing the simulations to existing data. From beginning to end, it curated my interest and applicable skills in modeling while also developing my knowledge of solar atmospheric dynamics. 

Professor Mark Rast (CU Boulder) and I began working on the previously described machine learning based Fourier filtering technique before my REU and is currently ongoing, where I will present my project as an honors thesis in the spring. This project has given me experience in a section of modeling quickly coming to prominence -- \textbf{machine learning}. Learning the statistics involved with modeling and machine learning as well as the possible applications of machine learning has increased both the types and number of projects I could work on. By pursuing a minor in statistics and building my knowledge of machine learning from the ground up, I have learned how to apply well-established knowledge from these fields while also having experience in learning from scratch without a class to guide me. \textbf{It is the experience in machine learning, neural networks, and statistics that I hope to bring to MSU}, along with the lessons learned and direct experience in my research experiences: presentation skills, writing, independence, question asking, and more. 

The REU experience, as well as the results of the project, has motivated me to apply to MSU to work on solar flare heating and energetics. The many open questions in flare energy partitioning, processes of heating, and the impacts on Earth are all areas of research I am interested in and are all areas MSU has strong research faculty in. Professor Dana Longcope's and Professor Jiong Qiu's work both interest me as potential research areas for graduate research. Montana State's physics program would also help me mature my understanding of the physical processes underlying solar phenomena. With leading experts in solar physics and rigorous courses available to graduate students, MSU would be a great fit for furthering my goals to better understand the sun and to become a researcher. I am interested in learning more about the governing equations and processes that dictate heating of plasmas, which requires more rigorous thermodynamics, electrodynamics, magnetodynamics, and fluid dynamics than available to an undergraduate. At MSU, I would be able to mature this understanding of magnetohydrodynamics through coursework and research with leading faculty.

Specifically, I am interested in modeling and machine learning as a method of better understanding solar flares in many ways. Modeling how magnetic reconnection related to flare energetics is important to understanding many observations of the sun, flares, CMEs, or coronal streamers. Ultimately, a better understanding of the energetics in the solar atmosphere helps understanding the coronal heating problem more rigorously. Machine learning is an important technique to understand and model these energetics, especially due to the complexity of the processes. I hope to bring my understanding of MHD and solar flares from my REU at MSU with me, along with my machine learning understanding, to the MSU physics department. A graduate experience at MSU would help me towards my eventual goal of being involved in research as a career. With this interest, and my interest in teaching, the most logical path would be towards a tenured position at a university. By combining my interests in teaching, research, and being involved in a community of science, I believe attending MSU's physics graduate program would be a critical step in reaching my career goals.

\textbf{Along with research, it is an equal, if not higher, priority to spend my career in the lecture hall, classroom, and community, helping students of all ages engage with science and technology.} We are in a critical time where science literacy is becoming significantly more important; by engaging students with science that MSU and the world is participating in, we can do our part in ensuring the future of the scientific community, and society as a whole. Underrepresented and underprivileged communities may have few opportunities to be exposed to the science a university or institute conducts, and it’s my goal to do anything in my power to ensure students understand what I have been told throughout my undergraduate career -- science is for everyone and can be done by anyone with curiosity and a drive to learn more about the world around them. My interest in science education research and a desire to help my community has placed teaching as a solidified part of my long-term career goals. MSU’s graduate school and programs would be a critical place for me to learn skills and gain experience in teaching while using the available resources to help the community.

Becoming involved with Montana's Space Grant Consortium is a prime opportunity to connect with the community and get people excited about science and the universe -- something any astronomer would take advantage of if given the opportunity. Working with the MTSGC on outreach initiatives is something I hope to do and I have done with the University of Colorado’s Sommers-Bausch Observatory in coordination with CU STARs (Science Technology and Astronomy Recruits). Star parties or open viewing sessions are something I have helped with at CU that has been very successful in engaging the community with physics and is a program I will continue in graduate school. MTSGC are great ways to connect with the community and engage with people of all ages and backgrounds who rarely, or never, have an opportunity to look through a telescope. As a graduate student, it would be my hope that I could transition into the current outreach initiatives MSU supports and continue to build on the experience I have gained in my undergraduate career to further my goal of engaging with local communities to better involve underrepresented groups in science. Seeing better engagement from underrepresented groups in physics and STEM as a whole is a career goal (and general life goal), and I believe MSU would be a great place to learn how I can better support my community while furthering my professional/research goals.  

My interest in graduate school is a shared sentiment with nearly everyone in the field: a deep desire to learn more about the universe and to share it with the rest of the world. MSU's programs will help me learn the skills I need to conduct research on solar physics, while I can, in return, bring the passion, desire to learn, and hard work faculty might want in a graduate student.

Thank you very much for taking your time reading this statement of purpose. I look forward to hearing from you soon and exploring the cosmos!
\end{document}
