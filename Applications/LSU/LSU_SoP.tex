\documentclass[12pt,letterpaper]{article}
\usepackage{fullpage}
\usepackage[top=2cm, bottom=2cm, left=2.5cm, right=2.5cm]{geometry}
\usepackage{amsmath,amsthm,amsfonts,amssymb,amscd}
\usepackage{lastpage}
\usepackage{enumerate}
\usepackage{fancyhdr}
\usepackage{mathrsfs}
\usepackage{xcolor}
\usepackage{graphicx}
\usepackage{listings}
\usepackage{hyperref}
\usepackage{float}
\usepackage{gensymb}
\usepackage[utf8]{inputenc}
\usepackage[english]{babel}
\usepackage{setspace}
\pagenumbering{gobble}


% Edit these as appropriate
\newcommand\hwnumber{1}                  % <-- homework number
\newcommand\NetIDa{Patrick McCreery}           % <-- NetID of person #1
\newcommand\NetIDb{108773641}           % <-- NetID of person #2 (Comment this line out for problem sets)
\setlength{\parindent}{2em}
\setlength{\parskip}{1em}
\renewcommand{\baselinestretch}{1}
\pagestyle{fancyplain}
\lhead{\NetIDa}
\rhead{Louisiana State University}
\headsep 2.0em

\begin{document}

Graduate school serves as a transition from undergraduate study to professional research positions, while learning more about specific areas of research a student is specifically interested in. For my case, I am interested in both solar physics and planetary science. Louisiana State University's physics department would serve well as a way for me to both develop my understanding of these subjects and to gain more robust skills in research. As it stands, I have been part of projects that have been shortened or interrupted because of COVID-19's impact on the United States, and I believe a rigorous program, like LSU's, would serve my research skills well in developing and maturing them. The faculty in the physics department is strong in the variety of their research areas. This benefits me, as I am interested in a broad range of study, and would allow me to find the right area of study to transition from undergraduate to professional research. 

As for specific research projects, I have been strongly interested in understanding the initial conditions of star-forming nebulae on the resulting planetary systems that may develop around the star, as well as the dynamics that occur between the nebula cloud and formed system. Having taken undergraduate courses on the subject, graduate study would be the natural step in understanding the processes and dynamics involved in forming planetary systems. To understand how planetary systems form, it is important to understand the existing exoplanet populations we are able to observe, and push the limits of what properties we are able to understand about exoplanet surveys. As a result, conducting surveys of exoplanet systems is important for the larger context of planet formation, and is a possible topic of interest for me in my graduate research. Developing current surveys is important in this endeavor and would be an interesting possible area of graduate research for me. Microlensing and transiting techniques are both important techniques in understanding exoplanet systems, and furthering our understanding of exoplanet systems via microlensing is especially important due to the expected launch of the Nancy Grace Roman Space Telescope, which will survey exoplanets using microlensing. Being part of a census that understands the conditions of habitability, and more largely the conditions impacting planet formation, are important to our understanding of planet systems as a whole, and a possible research project would involve bettering our understanding of planet properties using microlensing surveys. Roman brings promise of a thorough understanding of exoplanet systems; a graduate project of continuing to use current microlensing techniques to better our understanding of planets would be important when Roman eventually launches and sends back data. Having a robust understanding of how to use microlensing techniques and some of the possible properties we are able to get from the technique would be beneficial for using Roman data more efficiently.

Another topic of interest for me is understanding stars other than our own. Gaining more context about the larger populations of stars throughout the galaxy is important for understanding star formation, but also important for studying exoplanets and exoplanet systems. Metallicity of a star, for example, appears to be an important factor in understanding planetary systems and formation. The rates of planet occurrence is correlated and increases with metal-rich stars, meaning understanding the properties of stars is an important metric in understanding possible planetary systems around the stars. It would be an interesting project for my graduate work to be analyzing and characterizing a large survey of stars and making inferences about the planetary systems around them. Especially interesting would be investigating stars that have known exoplanets we have observed. If we are able to put together a survey of planet properties along with their host star's metallicity and other characteristics, we could possibly make inferences about what characteristics influence planet populations, but also how and why characteristics correlate with planet populations/characteristics. This involves understanding formational dynamics of both stars and protoplanetary disks, so the possibility of putting together a large study of planet properties and host star properties could be useful for many areas of study. 

Beyond research, graduate school is an important resource to begin to hone and develop lecture hall skills. As a teaching assistant, it would be my ultimate goal to learn from faculty how to most effectively lecture and organize a course. It has long been a passion of mine, since helping peers in high school with physics homework, to be a lead instructor in courses as part of my career. LSU's graduate program would be an ideal place to develop these skills and gain experience being in a classroom. A solid instructor is important for many classes and keeping students engaged with the material beyond the course's end; it is a goal to become an instructor that keeps students engaged with astronomy and physics. Beyond the lecture hall, becoming a quality instructor is important for engaging and promoting scientific literacy beyond the university setting and into the community around LSU. Events, such as star parties, observatory open houses, lectures, etc., are important for the visibility of the university, but also important in developing a more scientifically literate society. Our society's lack of understanding of science is more and more becoming an issue, with both widespread distrust and blind trust running rampant in our communities. Better informing society is not only beneficial to astronomy's longevity as a science, but also beneficial in educating the public. Graduate school is critical in developing communication skills and will be an important component of my graduate experience. Someday, I hope to give lectures to the public in settings that allow me to share the wonders astronomy and physics are uncovering. 

My interest in graduate school is many-fold, including expanding my knowledge of the universe, focusing on specific research areas, honing research skills, bettering my ability to explain science, and much more. LSU's graduate program would be an ideal setting for my professional development, as the department has a variety of world-class faculty that would be able to aid in my development, while I can provide assistance in their own goals. 

Thank you for reading this personal statement -- I appreciate your time.



\end{document}
