\documentclass[12pt,letterpaper]{article}
\usepackage{fullpage}
\usepackage[top=2cm, bottom=2.5cm, left=2.5cm, right=2cm]{geometry}
\usepackage{amsmath,amsthm,amsfonts,amssymb,amscd}
\usepackage{lastpage}
\usepackage{enumerate}
\usepackage{fancyhdr}
\usepackage{mathrsfs}
\usepackage{xcolor}
\usepackage{graphicx}
\usepackage{listings}
\usepackage{hyperref}
\usepackage{float}
\usepackage{gensymb}
\usepackage[utf8]{inputenc}
\usepackage[english]{babel}
\usepackage{setspace}
\pagenumbering{gobble}


% Edit these as appropriate
\newcommand\hwnumber{1}                  % <-- homework number
\newcommand\NetIDa{Patrick McCreery}           % <-- NetID of person #1
\newcommand\NetIDb{108773641}           % <-- NetID of person #2 (Comment this line out for problem sets)
\setlength{\parindent}{2em}
\setlength{\parskip}{1em}
\renewcommand{\baselinestretch}{1}
\pagestyle{fancyplain}
\lhead{\NetIDa}
\rhead{Louisiana State University}
\headsep 2.0em

\begin{document}

As a graduate student, I would like to make significant strides in actively helping underrepresented and underprivileged students in STEM realize their potential and further every student’s curiosity about the world around them. Entering LSU's graduate program would help me take a step forward in my career goals of helping communities. It is well documented that astronomy and physics have some of the poorest gender and race representation of any sciences, and this can be aided by encouraging and fostering curiosity in physics/astronomy among all students, which is something that I hope to do in graduate school and beyond. Recognizing the problem is one step, but actively solving the problem is another, and I believe I could be in a prime position to take strides in alleviating the problem if admitted. It is the duty of graduate students and those in more senior positions, who have been fortunate enough to be part of the scientific community, to help students who are less fortunate, whether that be from systematic discrimination, unfortunate circumstances, poor school funding, or many other possible circumstances. LSU has many programs that I would like to participate in to improve the scientific community through outreach. 

The K-12 district I attended for my whole life had a school board president who was a rocket scientist -- meaning I had a plethora of opportunities to get involved in STEM opportunities. These opportunities I had, like advanced calculus courses that were only taught because the district could afford to pay a qualified teacher to teach them, were important to developing my curiosity and interest in a STEM career that others may not have the luxury of being able to have. However, even with plenty of access to quality STEM courses, I still was not convinced that I had what it took to participate in research. This was because I couldn't see myself in the field; I had a conception of what a scientist was, and I didn't see myself in that image. Helping students envision themselves as a researcher is what I believe can help alleviate some of the demographic issues in astronomy/physics.

It’s critical for graduate students to understand their ability to help students in their community engage with science, even if it may not lead to a career in science for that student. Graduate students have the capacity to make impactful contributions in their communities and demonstrate to students that they are capable of participating in research and furthering their education. I took my astronomy class in high school with the wrestling coach who was contractually obligated to teach the course, while I knew there were students in private schools learning from educators with PhDs. If I had the opportunity to listen or talk to a graduate student in my time in middle or high school, it would have gone a long way to understanding that I had a place in science. This was only my experience as someone who comes from a good school district and was able to eventually see myself in a place to potentially do science, and as a graduate student, I want to ensure I can help students who were not as fortunate as me to understand that anyone with the curiosity and determination to succeed can fit into the scientific community. At best, students pursue a career in STEM, and at worst, graduate students can help better the scientific literacy of our community by participating in outreach initiatives.

Attending the University of Colorado at Boulder, I met Professor Erica Ellingson, who invited me to help CU STARs (Science, Technology, and Astronomy Recruits) in their goal to bring the sky to underrepresented students in STEM. I joined because I understood the opportunities that others may not have, and I wanted to help extend opportunities to students who may not see themselves in STEM. CU STARs is a group of passionate graduate and undergraduate students participating in a plethora of activities aimed at the goal of bringing astronomy, and STEM more broadly, to underrepresented groups who would typically not have the opportunity to learn about the sky or realize possible research opportunities available to them. We made in-class lessons on black holes, constellations, scales of the universe, and more to give to rural Colorado middle and high schools, then actually went and taught these courses. Interacting with students of all backgrounds was the most rewarding experience of my undergraduate career by far. These experiences have led to me attending graduate school; the experience I can gain, the greater opportunities to do outreach, as well learning how to improve my teaching skills are all reasons I am applying to graduate school. 

Part of my work with CU STARs -- and something that I hope to continue doing at LSU in graduate school -- is to bring the stars to people via star parties and open houses. At CU, campus-wide star parties we hosted brought hundreds of students (pre-pandemic) from every background and demographic to a wide variety of telescopes we set up to help them get a taste of the universe. It was rewarding to see the awe from students who had never seen Saturn's rings before, and furthering these events and hosting different kinds of events is why I am pursuing a graduate degree, where I will have more liberty and resources to do so. The Landolt Astronomical Observatory hosts public open houses that I would like to help with and connect with the community better. I have hosted virtual open houses for pandemic-times, but in-person open houses are important methods to connect with the wider community in Baton Rouge when safe.

Connecting with the community around LSU would do only good in improving the demographics in astronomy in the long term, but also improve the scientific literacy of the community, which is only beneficial. As a graduate student, I would like to join current outreach initiatives LSU has, such as those associated with LaSPACE and the Society of Physics Students, to be able to do my best to engage with students of all ages. It's critical that the public is exposed to science, and given the reach LSU has in Louisiana and beyond, the possibility of joining the physics department at LSU is exciting. Graduate students have unique opportunities to benefit the community around them, and is a driving reason I am applying to graduate school.

\textit{“Every kid starts out as a natural-born scientist, and then we beat it out of them. A few trickle through the system with their wonder and enthusiasm for science intact.”}

- Carl Sagan


\end{document}
