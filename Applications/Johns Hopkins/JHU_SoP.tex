\documentclass[11pt,letterpaper]{article}
\usepackage{fullpage}
\usepackage[top=2cm, bottom=2cm, left=2cm, right=2cm]{geometry}
\usepackage{amsmath,amsthm,amsfonts,amssymb,amscd}
\usepackage{lastpage}
\usepackage{enumerate}
\usepackage{fancyhdr}
\usepackage{mathrsfs}
\usepackage{xcolor}
\usepackage{graphicx}
\usepackage{listings}
\usepackage{float}
\usepackage{gensymb}
\usepackage[utf8]{inputenc}
\usepackage[english]{babel}
\usepackage{setspace}
\pagenumbering{gobble}
\usepackage[colorlinks = true,
            linkcolor = blue,
            urlcolor  = blue,
            citecolor = blue,
            anchorcolor = blue]{hyperref}

% Edit these as appropriate
\newcommand\hwnumber{1}                  % <-- homework number
\newcommand\NetIDa{Patrick McCreery}           % <-- NetID of person #1
\newcommand\NetIDb{108773641}           % <-- NetID of person #2 (Comment this line out for problem sets)
\setlength{\parindent}{2em}
\setlength{\parskip}{1em}
\renewcommand{\baselinestretch}{1}
\pagestyle{fancyplain}
\lhead{\NetIDa}
\rhead{Johns Hopkins University}
\headsep 2.0em

\begin{document}

At the University of Colorado at Boulder, I am working on an honors thesis project using machine learning to create synthetic images of the solar surface, and using these to train a neural network to perform a Fourier filter. This separates solar granulation from resonant spherical harmonics oscillating on the solar surface without the need for an observationally expensive time-series. The experience of modeling and simulating seemingly random physical phenomena has been one of the most rewarding experiences I have had, and is a driving factor in my desire to enter the Johns Hopkins physics and astronomy graduate program. This project, which began in 2020, led me to apply to REU programs related to simulations, and \textbf{I completed my 2021 (delayed from 2020) REU at Montana State University simulating solar flares}, investigating loop-top ridge heating post-flare, and learning about solar magnetodynamics. These research projects have taught me valuable skills in parameter searches and machine learning, for example, but have most importantly helped me understand that modeling and solar physics is what I wish to pursue in graduate research.
 
Professor Mark Rast (CU Boulder) and I began working on the previously described machine learning based Fourier filtering project before my REU and is currently ongoing, where I will present my project as an honors thesis in the spring. This project is supported by the Lab of Atmospheric and Space Physics through a scholarship awarded based on the quality of work proposal. More about this scholarship can be found on the \href{https://lasp.colorado.edu/home/about/scholarships-and-fellowships/the-charles-a-barth-scholarship-in-space-research/}{Charles A. Barth Scholarship webpage}. This project has provided me experience in a section of modeling quickly coming to prominence -- \textbf{machine learning}. Learning the statistics involved with modeling and machine learning in \textbf{Python} as well as the possible applications of machine learning has increased both the types and number of projects I could work on. By pursuing a minor in statistics and building my knowledge of machine learning from the ground up, I have learned how to apply well-established knowledge from these fields while also having experience in learning from scratch without a class to guide me. \textbf{It is the experience in machine learning, neural networks, and statistics that I hope to bring to JHU}, along with the lessons learned in my research experiences: presentation skills, writing, independence, question asking, and more.

\href{http://solar.physics.montana.edu/www/reu/2021/mccreery/}{My REU experience} at Montana State University helped mature my interest in modeling and gave me tangible experience – the results of which will be presented at the annual \textbf{AGU conference.} There have been questions about the origin of bright loop-top plasma ridge structures during solar flares, and my work at Montana State was running simulations to test if compressive magnetosonic shocks with drag could reproduce the observed structure, densities, and magnetic field retraction velocities. My work cast doubt on this idea, opening the field for alternative solutions or approaches. Throughout the REU, I spent significant time reading papers, scouring the \textbf{IDL} code of the simulations I was running, exploring different parameters that impacted heating after solar flares, and comparing the simulations to existing observations. From beginning to end, it curated my interest and applicable skills in modeling while also developing my knowledge of solar magnetic structure. 

However, solar physics is not where my heart necessarily lies, and as such I have used these experiences to develop skillsets in modeling while taking undergraduate and graduate courses in planetary interiors, surfaces, atmospheres, formations, dynamics, and more. During these classes, planetary system formation and exoplanet characterization have always been the most interesting topic to me due to the fascinating dynamics of extra-solar systems, the differences between our solar system and extra-solar systems, and the vast areas of research in the field of exoplanets (atmospheres, dynamics, surfaces, etc.). The combination of my desire to study planetary science and modeling has let me to JHU -- the place I believe most suitable to bring together my studies and work, while furthering my development as a scientist.

Prof. Kevin Schlaufman's work is most interesting to me in the department, especially in planet formation and analysis of star metallicities. The difference in solar nebula metallicities are critical in understanding planet formation and system dynamics; modeling and investigating the impacts of varying metallicities are exoplanet systems is a project I would like to approach. Furthermore, investigating exoplanets themselves in their current state are critical for understanding solar system formation, which is a topic that piqued my interest in undergraduate courses. Understanding the fundamentals of putting together a solar system helps us better understand the differences between planetary systems, and the potential habitability of systems. 

One of the most intriguing aspects of JHU's Institute for Data-intensive Engineering and Science. It took until my junior year with classes in data analysis in astronomy, a probability course, and a statistics course to realize I was deeply interested in data science -- one of the fastest growing areas in and out of academia. Unfortunately, I did not have the schedule space to add a statistics degree, but I have added an applied statistics minor. Astronomy is collecting more and more data to probe more and more complex problems in all areas of study via observation, and data science is becoming a critical aspect of being an astronomer. The applicability of data science to astronomy, and the world in general, make JHU's program one of the best for my career preparation to use large data sets for research applications in astronomy. Prof. Schlaufman's role in idies further interests me in his work. 

Throughout my undergraduate career, I have been taking courses at the graduate and undergraduate level to prepare myself for the physics and mathematics that will come with entering a graduate program. I have taken graduate level applied partial differential equations, as well as graduate level planetary surfaces and interiors, both of which I received an A in. These courses have taught me lessons in the mentality shift between undergraduate and graduate level studies, preparing me for this shift to a graduate program. Because of this, I am confident in my ability to adapt to graduate studies, but I am also prepared to take on the more difficult math and physics in a graduate program. I have taken a full year of classical mechanics, electrodynamics, and quantum mechanics, meaning I am confident in succeeding at JHU. I understand it will be difficult and take perseverance, but the graduate courses have helped me understand that I want to give my best effort.

\textbf{Along with research, it is an equal, if not higher, priority to spend my career in the lecture hall, classroom, and community, helping students of all ages engage with science and technology.} We are in a critical time where science literacy is becoming significantly more important; by engaging students with science that JHU and the world is participating in, we can do our part in ensuring the future of the scientific community, and society as a whole. Underrepresented and underprivileged communities may have few opportunities to be exposed to the science a university or institute conducts, and it’s my goal to do anything in my power to ensure students understand what I have been told throughout my undergraduate career -- science is for everyone and can be done by anyone with curiosity and a drive to learn more about the world around them. My interest in science education research and a desire to help my community has placed teaching as a solidified part of my long-term career goals. JHU’s program would be a critical place for me to learn skills and gain experience in teaching while using the available resources to help the community.

My interest in graduate school is a shared sentiment with nearly everyone in the field: a deep desire to learn more about the universe and to share it with the rest of the world. JHU’s programs will help me learn the skills I need to conduct research on planetary systems and exoplanets, while I can, in return, bring the passion, desire to learn, and hard work faculty might want in a graduate student. Beyond graduate studies, I hope to continue research and being as involved with teaching as possible. Being involved in academics at the university level would be the best course to achieve these goals, which is why entering a tenure-track position would be my post-graduate goal. JHU's program would be the prime place to aid in achieving this goal.

Thank you very much for taking your time reading this statement. I look forward to hearing from you soon and exploring the cosmos!
\end{document}
