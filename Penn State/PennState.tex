\documentclass[12pt,letterpaper]{article}
\usepackage{fullpage}
\usepackage[top=2cm, bottom=2.5cm, left=2.5cm, right=2cm]{geometry}
\usepackage{amsmath,amsthm,amsfonts,amssymb,amscd}
\usepackage{lastpage}
\usepackage{enumerate}
\usepackage{fancyhdr}
\usepackage{mathrsfs}
\usepackage{xcolor}
\usepackage{graphicx}
\usepackage{listings}
\usepackage{hyperref}
\usepackage{float}
\usepackage{gensymb}
\usepackage[utf8]{inputenc}
\usepackage[english]{babel}
\usepackage{setspace}
\pagenumbering{gobble}


% Edit these as appropriate
\newcommand\hwnumber{1}                  % <-- homework number
\newcommand\NetIDa{Patrick McCreery}           % <-- NetID of person #1
\newcommand\NetIDb{108773641}           % <-- NetID of person #2 (Comment this line out for problem sets)
\setlength{\parindent}{2em}
\setlength{\parskip}{1em}
\renewcommand{\baselinestretch}{1}
\pagestyle{fancyplain}
\lhead{\NetIDa}
\rhead{Pennsylvania State University}
\headsep 2.0em

\begin{document}

At the University of Colorado at Boulder, I am working on an honors thesis using machine learning to create synthetic images of the solar surface, and use these to train a neural network to perform a Fourier filter. This separates solar granulation from resonant spherical harmonics oscillating on the solar surface without the need for an observationally expensive time-series. The experience of modeling and simulating seemingly random physical phenomena has been one of the most rewarding experiences I have had, and is a driving factor in my desire to enter Penn State's astronomy and astrophysics graduate department. This project led me to apply to REU programs related to simulations, and I completed my 2021 (delayed from 2020) REU at Montana State University simulating solar flares and loop-top ridge features. These research projects have taught me valuable skills in parameter searches and machine learning, for example, but have also helped me understand that modeling is what I wish to pursue in graduate research and beyond.
 
My REU experience at Montana State University helped mature my interest in modeling and provided tangible experience – the results of which were presented in December at the annual \textbf{AGU conference.} There have been questions about the bright loop-top plasma ridge structures during solar flares, and my work at MSU surrounded running simulations to test if compressive magnetosonic shocks could reproduce the observed structure, densities, and magnetic field retraction velocities. My work cast doubt on this idea, opening the field for alternative solutions or approaches. Throughout the REU, I spent significant time reading papers, scouring the code of the simulations I was running, exploring different parameter spaces that impacted heating after solar flares, and comparing the simulations to observation. From beginning to end, it curated my interest and applicable skills in modeling while also developing my knowledge of solar atmospheric dynamics. 

Professor Mark Rast (CU Boulder) and I began working on the previously described machine learning based Fourier filtering technique before my REU and is currently ongoing, where I will present my project as an honors thesis in the spring. This project has given me experience in a section of modeling quickly coming to prominence -- \textbf{machine learning}. Learning the statistics involved with modeling and machine learning as well as the possible applications of machine learning has increased both the types and number of projects I could work on. By pursuing a minor in statistics and building my knowledge of machine learning from the ground up, I have learned how to apply well-established knowledge from these fields while also having experience in learning from scratch without a class to guide me. \textbf{It is the experience in machine learning, neural networks, and statistics that I hope to bring and develop at Penn State}, along with the skills learned in my research experiences: presentations, writing, independence, question asking, and more.

However, solar physics is not where my heart necessarily lies, and as such I have used these experiences to develop skillsets in modeling while taking undergraduate and graduate courses in planetary interiors, surfaces, atmospheres, formations, dynamics, and more. During these classes, \textbf{planetary formation and atmospheres} have always been the most interesting topics to me due to the fascinating dynamics within planet-forming disks (streaming instabilities, accretion, turbulence, etc.), the variety of observed planetary atmospheres and properties within our solar system and in other systems, as well as the variety of problems still facing the fields. The combination of my desire to study planetary science and modeling has let me to Penn State -- the place I believe most suitable to bring together my studies and work, while furthering my development as a scientist.

My interests in planetary formation, modeling, and habitability are well suited to the astronomy department at Penn State, with many leading faculty conducting research in this area. \textbf{Most interesting to me are the works of Dr. Rebekah Dawson and Dr. Ian Czekala}. Dr. Dawson's work regarding the origins of and dynamics leading to the formation of hot Jupiters, as well as investigating parameters of hot Jupiters, intrigue me due to the surprisingly close orbits to their host star and interesting dynamics and problems involving their formation and migration. Dr. Dawson's work in understanding the origins of hot Jupiters interests me and would be an ideal group to work in for my graduate studies. Developing my understanding of formational dynamics, which I have had courses in, as well as maturing my skills in modeling are both end goals of mine for my graduate studies. Dr. Czekala's work also particularly interests me because of the close connection between computation, data science, and planet formation, which are all areas of research I hope to conduct research in during my career. Furthermore, I would like to gain an understanding about the methods of analyzing and interpreting observations from ALMA to investigate chemical structures in protoplanetary disks.

Penn State has many different programs and resources that make their astronomy department the most intriguing and exciting for me to apply to. Chief among these reasons are the Center for Astrostatistics and the Center for Exoplanets \& Habitable Worlds. 2 years into my undergraduate career, after taking statistics courses, a scientific programming course, and a data analysis in astronomy course, I understood my interest in statistics, probability, and data analysis. However, being 2 years into my degree and for financial reasons, I could not add a major in statistics; instead, I added courses in statistics for a minor, graded for the aforementioned data analysis course, and found online resources to teach myself data analysis in Python and R. Now, I find myself wanting to apply this knowledge in graduate school. \textbf{Penn State has the unique and well established Center of Astrostatistics where I would be excited to work and learn.} As missions and observations are growing more complex and data heavy, there are statistical issues that prompt the necessary development of statistical solutions in an astrophysical/observational context. This desire to solve problems in this area, whose solutions are applicable far beyond astronomy, motivated me to apply to Penn State to better my understanding and help develop solutions to the many problems facing the field of astrostatistics. 

\textbf{Along with research, it is an equal, if not higher, priority to spend my career in the lecture hall, classroom, and community, helping students of all ages engage with science and technology.} We are in a critical time where science literacy is becoming significantly more important; by engaging students with science that Penn State and the world is participating in, we can do our part in ensuring the future of the scientific community, and society as a whole. Underrepresented and underprivileged communities may have few opportunities to be exposed to the science a university or institute conducts, and it’s my goal to do anything in my power to ensure students understand what I have been told throughout my undergraduate career -- science is for everyone and can be done by anyone with curiosity and a drive to learn more about the world around them. My interest in science education research and a desire to help my community has placed teaching as a solidified part of my long-term career goals. Penn State's graduate school and programs would be a critical place for me to learn skills and gain experience in teaching while using the available resources to help the community.

The Astronomy and Astrophysics Department and Astronomy Club's weekly stargazing sessions are a prime opportunity to connect with the community and get people excited about science and the universe -- something any astronomer would take advantage of if given the opportunity. Working with the department on outreach initiatives is something I hope to do and I have done with the University of Colorado’s Sommers-Bausch Observatory in coordination with CU STARs (Science Technology and Astronomy Recruits).  

During my time with CU STARs, the most rewarding initiatives were those that went to classrooms to engage K-12 students with science and encourage them to stay engaged with their school and beyond. Giving lessons to a class of students (both in person and virtually) were important to my understanding of the issues facing underrepresentation in astronomy, as well as the importance in actively supporting underprivileged students. The planetarium at Penn State, AstroFest, and the field trip opportunities are all important initiatives to engage the community with science. \textbf{As a graduate student, it would be my hope that I could transition into the current outreach initiatives Penn State supports and continue to build on the experience I have gained in my undergraduate career.}

Penn State's graduate program is a leading program in the nation, and the research and outreach programs at Penn State would be important in my own development as a productive astronomy community member.

Thank you very much for taking your time reading this statement of purpose. I look forward to hearing from you soon and exploring the cosmos!


\end{document}
