\documentclass{article}
\usepackage{geometry}
\usepackage{dirtytalk}
\usepackage{amssymb,graphicx}
\usepackage{hyperref}
\hypersetup{
    colorlinks=true,
    linkcolor=blue,
    filecolor=magenta,      
    urlcolor=blue}
\newcommand{\divider}{\vskip-2pt\hrule\vskip4pt}
\geometry{margin=.3in}
\setcounter{secnumdepth}{0}
\setlength{\parindent}{0pt}
\usepackage{tabularx}


\begin{document}
\begin{center}
	\huge\textbf{Patrick McCreery}
\end{center}
\begin{tabularx}{\textwidth}{
    @{\hspace{}}% Space for left bullet
    >{\leavevmode\llap{}\raggedright}% Left bullet + formatting of column
    X% Left column specification
    @{\quad\hspace{2.5cm}}% Space between columns + right bullet space
    >{\leavevmode\llap{}\raggedright\arraybackslash}% Right bullet + formatting of column
    X% Right column specification
    @{}% No column space on right
  }
  4930 Meredith Way & (720) 326-6436 \\
  BLDG. 04 APT. 304 & patrick.mccreery@colorado.edu \\
  Boulder, CO 80303 & REU Website: \href{http://solar.physics.montana.edu/reu/2021/mccreery/}{http://solar.physics.montana.edu/reu/2021/mccreery/}
\end{tabularx}

\section{Research Interests}
\divider


\begin{tabularx}{\textwidth}{
    @{\hspace{}}% Space for left bullet
    >{\leavevmode\llap{}\raggedright}% Left bullet + formatting of column
    X% Left column specification
    @{\quad\hspace{2.5cm}}% Space between columns + right bullet space
    >{\leavevmode\llap{}\raggedright\arraybackslash}% Right bullet + formatting of column
    X% Right column specification
    @{}% No column space on right
  }
  Planetary system formation & Solar flares and magnetic reconnection post-flare \\
  Planetary atmospheres (formations, dynamics, escape) & Solar variability \& terrestrial impacts \\
  Exoplanet atmospheres and detection & Star formation \\
  Machine learning applications in general & Science education and outreach\\
  \end{tabularx}


\section{Education}
\divider
\textbf{University of Colorado, Denver} 2017-2018

\quad Dual-Enrollment

\textbf{University of Colorado, Boulder} 2018-2022

\quad B.A., Astronomy, Astrophysics and Planetary Science emphasis, May 2022 (\textit{Expected})

\quad \quad Minor, Applied Mathematics, Statistical emphasis

\quad \textbf{Cumulative GPA, Astronomy/Physics GPA}: 3.99/4.0, 4.0/4.0



\quad Thesis (in progress): \textit{Fourier Filtering using Machine Learning}

\section{Computer and Technical Skills}
\divider

{\bf Computer Languages} \hspace{10pt} { Python, IDL, R, HTML, Basic JavaScript}\\
{\bf Software \& Tools} \hspace{30pt} { \LaTeX, Mathematica, MATLAB, Microsoft Suite}\\

\section{Research Positions and Projects}
\divider

\subsection{Resolving Source Solar Acoustic Oscillations - Lab of Atmospheric and Space Physics \\ (April 2020-present)}

Building upon the work of \href{https://arxiv.org/abs/1811.08944}{McClure, Rast, and Pillet}, I used Fourier Transforms to resolve solar acoustic oscillations and separate these from the surface granulation. Upon resolving and separating the p-modes and granulation, I am now training a machine to separate these two components without requiring a time series. Ongoing project with a hopeful publication and honors thesis upon completion. 

\subsection{Simulating Solar Flares - Montana State University (summer 2021, REU)}

As part of an REU Program, I simulated solar flares post-magnetic reconnection using magnetosonic shocks. Attempted to resolve magnetic field retraction velocity inconsistencies of previous work. The observed and simulated retraction velocities were inconsistent in previous work, however, my work introducing aerodynamic drag to the simulation could not resolve these inconsistencies, opening opportunity for alternative solutions to heating of loop-top ridges post-reconnection. 

\subsection{Detecting and Characterizing Trans-Neptunian Objects (Research Methods in Astronomy)}

As part of the Research Methods in Astronomy course taken in spring of 2020, I took part in helping resolve, analyze, and characterize the shapes of occultation targets observed by RECON (Research and Education Collaborative Occultation Network). The organization had data from occultation events of objects in the outer solar system to analyze. We used Python and IDL to identify the target star that a TNO may occult and analyze if the object occulted the star. The analysis was then used to characterize the shape of the TNO. 

\subsection{Modeling Plume Deposits on Europa (Graduate Course Project)}

As part of the graduate-level Planetary Surfaces and Interiors course, I designed, completed, and presented an open-ended project regarding plume deposits on Europa. The project investigated whether plumes from surface vents on Europa could plausibly deposit an icy layer of material detectable by Europa Clipper's thermal imager, E-THEMIS. Involved understanding ejection from deep-sources, particle size distributions, thermodynamics of fine-grain material, and different models of material build-up. Project provided plausible timescales for how long plume deposits take to build and found it likely E-THEMIS could detect deposits.

\section{Relevant Coursework}
\divider
\begin{tabularx}{\textwidth}{
    @{\hspace{}}% Space for left bullet
    >{\leavevmode\llap{}\raggedright}% Left bullet + formatting of column
    X% Left column specification
    @{\quad\hspace{2.5cm}}% Space between columns + right bullet space
    >{\leavevmode\llap{}\raggedright\arraybackslash}% Right bullet + formatting of column
    X% Right column specification
    @{}% No column space on right
  }
  \textbf{Physics} & \textbf{Mathematics} \\
  Classical Mechanics I/II & Calculus, Differential Equations Sequences\\
  Electricity and Magnetism I/II & Matrix Methods and Applications\\
  Quantum Mechanics I/II & \textit{Fourier Series and Boundary Value Problems (PDEs)}$^*$\\
  Thermodynamics and Statistical Mechanics$^\dagger$ & Applied Probability \& Statistics Sequences$^\dagger$\\
  Experimental Physics I/II & Matrix Methods and Linear Algebra\\
  \textbf{Astronomy Courses} & \\
  Planetary Atmospheres and Geology & Astrophysics I \\
  Plasma and Space Physics$^\dagger$  & Data Analysis and Research Methods in Astronomy  \\
  \textit{Planetary Surfaces and Interiors}$^*$ &  Mountain Geography\\
  \\
  $^\dagger$ indicates current enrollment \\
  $^*$ indicates a graduate level course
\end{tabularx}

\section{Leadership, Volunteer, and Other Positions}
\divider

\subsection{CU STARs (2019-)}
The Astrophysical and Planetary Sciences describe CU-STARs (Science, Technology and Astronomy Recruits) as \say{a University of Colorado program to support students from all backgrounds interested in space.} I began loosely participating in 2019 as part of a program to host a campus wide star party, where I helped put together and use telescopes at the star party, as well as explain the object(s) in view and answer questions that attendees might have. In 2020, I began attending meetings more regularly, helping plan, develop, and present topics in astronomy to underrepresented and rural schools in Colorado.

\subsection{Resident Advisor (2019-2020)}
For the 2019-2020 school year, I worked as a resident advisor in the CU dorms, helping incoming freshman transition into college life. Work duties involved keeping in touch with each resident, putting on programs that kept students engaged in the communities, and being \say{on-call} for potential issues and crises that may arise in the residence halls. I left the position in 2020 due to concerns regarding COVID-19.

\subsection{Applied Mathematics Writer (2019-)}
In 2019, I joined the Applied Mathematics Department at CU, writing columns and news articles that post to the applied math website (amath.colorado.edu). I created and edited the yearly newsletter, maintained portions of the applied math website, and helped the IT specialist, office manager, and undergraduate coordinator. 

\subsection{CubeSat Project Member -- Colorado Space Grant Consortium (2019)}
For the spring semester of 2019, the Colorado Space Grant Consortium brought together students from all STEM fields to work on CubeSat projects that would ascend through Earth's atmosphere on a weather ballon. As part of the project, we designed, built, tested, and launched a project relevant to an ascending ballon. The team chose a project analyzing pressure, temperature, and solar flux with altitude in the atmosphere. I helped conceptualize the project, build the structure of the cube, test the GoPro fixed to the cube, and analyze the collected data. 

\subsection{Department of Astrophysics and Planetary Science Grader}
During the summer of 2019, I graded for an introductory astronomy course, then in the fall of 2021, I graded for a course in data analysis in astronomy.

\section{Presentations and Posters}
\divider

\textbf{McCreery, P.} \& Longcope, D. (2021). Thin Flux Tube Retraction Following Reconnection as a Model for the Observed Hot, Loop-top Ridge Structure in a Two-ribbon Flare. \textit{In preparation for Fall AGU conference}.

\pagebreak

\section{Awards and Scholarships}
\divider
\subsection{Charles A. Barth Scholarship (2021)}

Research scholarship promoting student projects in the Lab for Atmospheric Space Physics. Given based on quality of classwork, previous work, and proposed research work. 

\subsection{Theodore Snow Undergraduate Scholarship (2021)}

Scholarship given to undergraduates in the Astrophysical and Planetary Science department \say{in order to recognize the student's academic performance in coursework and research}
\subsection{Dean's List (2018 - present)}
\subsection{President James H. Baker Scholarship (2018 - present)}


\section{References}
\divider
\begin{tabularx}{\textwidth}{
    @{\hspace{}}% Space for left bullet
    >{\leavevmode\llap{}\raggedright}% Left bullet + formatting of column
    X% Left column specification
    @{\quad\hspace{2.5cm}}% Space between columns + right bullet space
    >{\leavevmode\llap{}\raggedright\arraybackslash}% Right bullet + formatting of column
    X% Right column specification
    @{}% No column space on right
  }
  \textbf{Professor Mark Rast} & \textbf{Professor David Brain} \\
  
Research Advisor & Academic Reference\\ 

Phone: (303) 735-1038 & Phone: (303) 735-5606\\

Email: mark.rast@lasp.colorado.edu & Email: david.brain@colorado.edu\\


  
    \textbf{Professor Erica Ellingson} & \textbf{Assistant Professor Paul Hayne} \\
    Outreach Reference & Academic Reference/Undergraduate Mentor\\ 

Phone: (303) 492-6610 & Phone: (303) 735-6399\\

Email: erica.ellingson@colorado.edu & Email: Paul.Hayne@Colorado.edu\\

      \textbf{Visiting Professor Ivan Milic} & \textbf{} \\
      Academic Reference & \\ 
      
	  Email: ivan.milic@colorado.edu & \\

\end{tabularx}

 

\end{document}