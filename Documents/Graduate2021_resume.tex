\documentclass{article}
\usepackage{geometry}
\usepackage{dirtytalk}
\usepackage{amssymb,graphicx}
\usepackage{hyperref}
\hypersetup{
    colorlinks=true,
    linkcolor=blue,
    filecolor=magenta,      
    urlcolor=blue}
\newcommand{\divider}{\vskip-2pt\hrule\vskip4pt}
\geometry{margin=.3in}
\setcounter{secnumdepth}{0}
\setlength{\parindent}{0pt}

\begin{document}
\begin{center}
	\huge\textbf{Patrick McCreery}
\end{center}

\begin{center}
	(720) 326 - 6436 $\diamond$ Patrick.McCreery@colorado.edu
\end{center}


\section{Education Summary}
\divider
{\bf University of Colorado, Denver} \hfill {\em August 2017 - May 2018}

\quad \quad \hfill Cumulative GPA: 4.0

{\bf University of Colorado, Boulder} \hfill {\em August 2018 - May 2022}

\quad \quad Undergraduate of Astrophysics (Planetary Science Track) \hfill Cumulative GPA: 3.99

\quad \quad \quad Minor: Applied Mathematics w/ Statistical Emphasis \hfill Total Credit Hours Passed: 126

\quad \quad Department of Astrophysical and Planetary Sciences\\
\section{Computer and Technical Skills}
\divider

{\bf Computer Languages} \hspace{10pt} { Python, IDL, HTML, Basic JavaScript}\\
{\bf Software \& Tools} \hspace{30pt} { \LaTeX, Mathematica, MATLAB, Microsoft Suite}\\

\section{Relevant Coursework (SHOULD THIS BE HERE w/ TRANSCRIPT?)}

\divider


Planetary Atmospheres and Geology (/) \hfill Astrophysics I (/)

Applied Probability () \hfill Applied Statistical Methods I/II (/)

Matrix Methods and Applications () \hfill Fourier Series and Boundary Value Problems (PDEs) ()

Plasma and Space Physics () \hfill Data Analysis and Research Methods in Astronomy (/)

Classical Mechanics I/II (/), Electricity and Magnetism I/II (/), and Quantum Mechanics I/II (/)

\section{Research and Project Experience}
\divider

\subsection{Resolving Source Solar Acoustic Oscillations - Lab of Atmospheric and Space Physics \\ (April 2020-present)}

Building upon the work of \href{https://arxiv.org/abs/1811.08944}{McClure, Rast, and Pillet}, I used Fourier Transforms to resolve solar acoustic oscillations and separate these from the surface granulation. Upon resolving and separating the p-modes and granulation, I am now training a machine to separate these two components without requiring a time series. Ongoing project with a hopeful publication and honors thesis upon completion. 

\subsection{Simulating Solar Flares - Montana State University (summer 2021)}

Simulated solar flares post-magentic reconnection using magnetosonic shocks. Attempted to resolve magnetic field retraction velocity inconsistencies of previous work. The observed and simulated retraction velocities were inconsistent in previous work, however the work introducing aerodynamic drag to the simulation could not resolve these inconsistencies, opening opportunity for alternative solutions. 

\section{Presentations}
\divider

AGU

\section{Awards and Scholarship}
\divider

\subsection{Charles A. Barth Scholarship}

Research scholarship promoting student projects in the Lab for Atmospheric Space Physics. Given based on quality of classwork, previous work, and proposed work. 

\subsection{Theodore Snow Undergraduate Scholarship (2021 - 2022 Academic Year)}

Scholarship given to undergraduates in the Astrophysical and Planetary Science department \say{in order to recognize the student's academic performance in coursework and research}

\end{document}