\documentclass[11pt,letterpaper]{article}
\usepackage{fullpage}
\usepackage[top=1.5cm, bottom=2cm, left=2.0cm, right=2.0cm]{geometry}
\usepackage{amsmath,amsthm,amsfonts,amssymb,amscd}
\usepackage{lastpage}
\usepackage{enumerate}
\usepackage{fancyhdr}
\usepackage{mathrsfs}
\usepackage{xcolor}
\usepackage{graphicx}
\usepackage{listings}
\usepackage{hyperref}
\usepackage{float}
\usepackage{gensymb}
\usepackage[utf8]{inputenc}
\usepackage[english]{babel}
\usepackage{setspace}
\pagenumbering{gobble}


% Edit these as appropriate
\newcommand\hwnumber{1}                  % <-- homework number
\newcommand\NetIDa{Patrick McCreery}           % <-- NetID of person #1
\newcommand\NetIDb{108773641}           % <-- NetID of person #2 (Comment this line out for problem sets)
\setlength{\parindent}{2em}
\setlength{\parskip}{1em}
\renewcommand{\baselinestretch}{1}
\pagestyle{fancyplain}
\lhead{\NetIDa}
\rhead{University of Arizona}
\headsep 2.0em

\begin{document}

Astronomy as a science is a unique human endeavor — bringing together the hard science of observation with the theory of philosophy. Astronomy not only probes the possibilities of the universe, but tests our resolve in understanding the consequences of being alive, exploring the universe, and considering if the universe is something to explore or something to observe from our backyard. This combination has pushed me to earn an undergraduate degree in astronomy/astrophysics and now a graduate degree. I’m driven to further our understanding of the great unknown, but also intrigued by the philosophical discussions to be had about exploring the universe and possible consequences of discoveries -- like the possible habitability of other solar systems. My understanding of the universe has breadth because of my undergraduate degree; however, now I look forward to earning a graduate degree and gaining depth in understanding of the topics I find most interesting by using the skillsets and knowledge I have gathered in my undergraduate experience. The topics that most interest me lie primarily in solar physics and planetary science. Though my work to date has been exclusively in solar physics, I have been taking steps to possibly begin working in planetary science by taking graduate level courses and working on class projects tailored to planetary science. The University of Arizona's astronomy program's strong faculty, varied research areas, and the program's mission is why I am applying to UA to further my interest and  goals in astronomy.

At the University of Colorado at Boulder, I am working on an honors thesis project using machine learning to create synthetic images of the solar surface, and use these to train a neural network to perform a Fourier filter. This separates solar granulation from resonant spherical harmonics oscillating on the solar surface without the need for an observationally expensive time-series. The experience of modeling and simulating seemingly random physical phenomena has been one of the most rewarding experiences I have had, and is a driving factor in my desire to enter UA's Department of Astronomy. This project, which began in 2020, led me to apply to REU programs related to simulations, and I completed my 2021 (delayed from 2020) REU at Montana State University simulating solar flares and loop-top ridge heating post-flare. These research projects have taught me valuable skills in parameter searches and machine learning, for example, but have most importantly helped me understand that modeling is what I wish to pursue in graduate research.
 
My REU experience at Montana State University, with advisor Professor Dana Longcope, helped mature my interest in modeling and gave me tangible experience – the results of which will be presented in December at the annual \textbf{AGU conference.} There has been questions about the bright loop-top plasma ridge structures during solar flares, and my work at Montana State involved me running simulations to test if compressive slow magnetosonic shocks (SMS) with drag could reproduce the observed structure, densities, and magnetic field retraction velocities. I performed a parameter search to attempt to reproduce observed properties of the loop-top plasma. My work cast doubt on the SMS hypothesis, but opened the field for alternative solutions or approaches to the problem. Throughout the REU, I spent significant time reading papers, scouring the code of the simulations I was running, exploring different parameters that impacted heating after solar flares, and comparing the simulations to existing data. From beginning to end, it curated my interest and applicable skills in modeling while also developing my knowledge of solar atmospheric dynamics. 

Professor Mark Rast (CU Boulder) and I began working on the previously described machine learning based Fourier filtering technique before my REU and is currently ongoing, where I will present my project as an honors thesis in the spring. This project has given me experience in a section of modeling quickly coming to prominence -- \textbf{machine learning}. Learning the statistics involved with modeling and machine learning as well as the possible applications of machine learning has increased both the types and number of projects I could work on. By pursuing a minor in statistics and building my knowledge of machine learning from the ground up, I have learned how to apply well-established knowledge from these fields while also having experience in learning from scratch without a class to guide me. \textbf{It is the experience in machine learning, neural networks, and statistics that I hope to bring to UA}, along with the lessons learned in my research experiences: presentation skills, writing, independence, question asking, and more.

However, \textbf{solar physics is not where my heart necessarily lies}, and as such I have used these experiences to develop skillsets in modeling while taking undergraduate and graduate courses in planetary interiors, surfaces, atmospheres, formations, dynamics, and more to be prepared for work in planetary science. During these classes, planetary atmospheres and solar system formation were the most interesting topics to me due to the many interesting modeling and observational techniques in these fields. As well, the possibilities of where modeling can improve in these areas is what I hope to pursue in graduate study. The combination of my desire to study planetary science and modeling has let me to UA -- the place I believe most suitable to bring together my studies and work, while furthering my development as a scientist.

UA's astronomy department is conducting a plethora of research I am interested in, like exoplanet detection, exoplanet characterization, solar system formation and dynamics, as well as star formation. Specifically, I am most interested in Dr. Kaitlin Kratter's work in formations of stellar and planetary systems. Protoplanetary disks have many unknown dynamics occurring that have yet to be understood, including the instabilities and turbulence in the disk that Dr. Kratter studies. As well, the formational dynamics of Pluto's system piques my interest, especially in the modeling the tidal dynamics in the Pluto-Charon binary. Dr. Kratter's work in analysis and computation is at the intersection of my interest in modeling and system formations/dyanamics, and I believe it is in these areas I can contribute the most.

Furthermore, I am interested in Dr. Apai's work in habitability of planets and extra-solar systems. The photometry and spectroscopy involved in the Cloud Atlas program and the characterization of exoplanet atmospheres with the ACCESS Survey are both interesting to me, including the search for possible life signatures in these exoplanet atmospheres. Along with Dr. Kratter and Dr. Apai, Dr. Eisner's work with understanding planet formation around young stars combines both of my interests in solar physics and planetary science. 

\textbf{Along with research, it is an equal, if not higher, priority to spend my career in the lecture hall, classroom, and community, helping students of all ages engage with science and technology.} We are in a critical time where science literacy is becoming significantly more important; by engaging students with science that UA and the world is participating in, we can do our part in ensuring the future of the scientific community, and society as a whole. Underrepresented and underprivileged communities may have few opportunities to be exposed to the science a university or institute conducts, and it’s my goal to do anything in my power to ensure students understand what I have been told throughout my undergraduate career -- science is for everyone and can be done by anyone with curiosity and a drive to learn more about the world around them. My interest in science education research and a desire to help my community has placed teaching as a solidified part of my long-term career goals. UA’s graduate school and programs would be a critical place for me to learn skills and gain experience in teaching while using the available resources to help the community. UA has accomplished lecturers, and I hope to be able to learn about education at a top astronomy department like UA's.

After graduate study, I hope to be continually active in research and education, which leads me to wanting to become a tenured-track professor in the long-term. My interest in graduate school is an important step towards my long-term goals and is driven by shared sentiment with nearly everyone in the field: a deep desire to learn more about the universe and to share it with the rest of the world. UA’s programs will help me learn the skills I need to conduct research in modeling planetary systems, while I can, in return, bring the passion, desire to learn, and hard work faculty might want in a graduate student.

Thank you very much for taking your time reading this statement of purpose. I look forward to hearing from you soon and exploring the cosmos!
\end{document}
